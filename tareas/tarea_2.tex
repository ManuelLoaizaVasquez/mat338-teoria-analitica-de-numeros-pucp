\documentclass{article}
\usepackage[utf8]{inputenc}
\usepackage{amsfonts,latexsym,amsthm,amssymb,amsmath,amscd,euscript}
\usepackage{mathtools}
\usepackage{framed}
% Descomentar fullpage cuando se quiera utilizar menos margen horizontal
%\usepackage{fullpage}
\usepackage{hyperref}
    \hypersetup{colorlinks=true,citecolor=blue,urlcolor =black,linkbordercolor={1 0 0}}

\newenvironment{statement}[1]{\smallskip\noindent\color[rgb]{1.00,0.00,0.50} {\bf #1.}}{}
\allowdisplaybreaks[1]

\DeclarePairedDelimiter\ceil{\lceil}{\rceil}
\DeclarePairedDelimiter\floor{\lfloor}{\rfloor}

% Comandos para teoremas, definiciones, ejemplos, lemas, etc. para sus respectivos body types.
\renewcommand*{\proofname}{Prueba}
\renewcommand{\contentsname}{Contenido}

\newtheorem{theorem}{Teorema}
\newtheorem*{proposition}{Proposici\'on}
\newtheorem{lemma}[theorem]{Lema}
\newtheorem{corollary}[theorem]{Corolario}
\newtheorem{conjecture}[theorem]{Conjetura}
\newtheorem*{postulate}{Postulado}
\theoremstyle{definition}
\newtheorem{defn}[theorem]{Definici\'on}
\newtheorem{example}[theorem]{Ejemplo}

\theoremstyle{remark}
\newtheorem*{remark}{Observaci\'on}
\newtheorem*{notation}{Notaci\'on}
\newtheorem*{note}{Nota}

% Define tus comandos para hacer la vida m�s f�cil.
\newcommand{\BR}{\mathbb R}
\newcommand{\BC}{\mathbb C}
\newcommand{\BF}{\mathbb F}
\newcommand{\BQ}{\mathbb Q}
\newcommand{\BZ}{\mathbb Z}
\newcommand{\BN}{\mathbb N}

\title{MAT338 Teor\'ia Anal\'itica de N\'umeros}
\author{Manuel Loaiza Vasquez}
\date{Ciclo 2020-1}

\begin{document}

\maketitle

\vspace*{-0.25in}
\centerline{Pontificia Universidad Cat\'olica del Per\'u}
\centerline{Lima, Per\'u}
\centerline{\href{mailto:manuel.loaiza@pucp.edu.pe}{{\tt manuel.loaiza@pucp.edu.pe}}}
\vspace*{0.15in}

\begin{framed}
Segunda tarea del curso de T\'opicos de An\'alisis de la Especialidad de Matem\'aticas dictado en la Facultad de Ciencias e Ingenier\'ia en la Pontificia Universidad Cat\'olica del Per\'u (PUCP) por Alfredo Poirier Schmitz en el ciclo 2020-1.
\end{framed}

\begin{statement}{1}
Conseguir estimados de orden $\cfrac{1}{n^2}$ por arriba y por abajo para
$$\sum_{k \leq n} \frac{1}{k} - \ln\left(n + \frac{1}{2} - \gamma\right).$$
\end{statement}
Primero hallemos el l\'imite
$$\lim_{n \to \infty} \left(\sum_{k = 1} ^ n \frac{1}{k} - \ln(n + r) \right)$$
para valores positivos de $r$.
\begin{align*}
\sum_{k = 1} ^ n \frac{1}{k} - \ln(n + r) &= \sum_{k = 1} ^ n \frac{1}{k} -\ln(n) + \ln(n) - \ln(n + r)\\
&= \sum_{k = 1} ^ n \frac{1}{k} - \ln(n) + \ln\left(\frac{n}{n + r}\right)\\
&= \sum_{k = 1} ^ n \frac{1}{k} - \ln(n) - \ln\left(\frac{n + r}{n}\right)\\
&= \sum_{k = 1} ^ n \frac{1}{k} - \ln(n) - \ln\left(1 + \frac{r}{n}\right).\\
\end{align*}
Aplicamos aritm\'etica de l\'imites al resultado anterior
\begin{align*}
\lim_{n \to \infty} \left(\sum_{k = 1} ^ n \frac{1}{k} - \ln(n + r)\right) &=
\lim_{n \to \infty} \left(\sum_{k = 1} ^ n \frac{1}{k} - \ln(n)\right) - \lim_{n \to \infty} \left(\ln\left(1 + \frac{r}{n}\right)\right)\\
&= \gamma - 0\\
&= \gamma.
\end{align*}
Asimismo, del corolario $21.2$ tenemos lo la siguientes igualdad
\begin{align*}
\sum_{k = 2} ^ n \frac{1}{k} &= \int_1^n \frac{dt}{t} + \int_1^n (t - \floor{t})\left(-\frac{1}{t^2}\right) dt\\
&= \int_1^n \frac{dt}{t} - \int_1^n \frac{t - \floor{t}}{t^2} dt\\
&= \ln(n) - \int_1^n \frac{t - \floor{t}}{t^2} dt.
\end{align*}
Sumamos uno a cada lado de la igualdad y obtenemos el siguiente resultado
\begin{align*}
1 + \sum_{k = 2} ^ n \frac{1}{k} &= 1 + \ln(n) - \int_1^n \frac{t - \floor{t}}{t^2} dt\\
\sum_{k = 1} ^ n \frac{1}{k} &= 1 + \ln(n) - \int_1^n \frac{t - \floor{t}}{t^2} dt.
\end{align*}
Despejamos la diferencia entre la serie arm\'onica y el logaritmo natural para aplicar l\'imites
\begin{align*}
\sum_{k = 1} ^ n \frac{1}{k} - \ln(n) &= 1 - \int_1^n \frac{t - \floor{t}}{t^2} dt\\
\lim_{n \to \infty} \left(\sum_{k = 1} ^ n \frac{1}{k} - \ln(n)\right) &= 1 - \int_1^\infty \frac{t - \floor{t}}{t^2} dt.
\end{align*}
As\'i obtenemos que la constante de Euler es igual a lo calculado previamente
$$\gamma = 1 - \int_1^\infty \frac{t - \floor{t}}{t^2} dt.$$
Combinaremos los resultados anteriores utilizando $\ln(n + r)$ con $r = \cfrac{1}{2}$
\begin{align*}
\sum_{k = 2} ^ n \frac{1}{k} &= \int_1^n \frac{dt}{t} + \int_1^n (t - \floor{t})\left(-\frac{1}{t^2}\right) dt\\
&= \int_1^n \frac{dt}{t} + \int_n^{n + \frac{1}{2}} \frac{dt}{t} - \int_n^{n + \frac{1}{2}}\frac{dt}{t} - \int_1^n \frac{t - \floor{t}}{t^2} dt\\
&= \int_1^{n + \frac{1}{2}} \frac{dt}{t} - \int_n^{n + \frac{1}{2}}\frac{dt}{t} + - \int_1^n \frac{t - \floor{t}}{t^2} dt\\
&= \ln\left(n + \frac{1}{2}\right) - \int_n^{n + \frac{1}{2}}\frac{dt}{t} - \int_1^n \frac{t - \floor{t}}{t^2} dt.
\end{align*}
Sumamos uno a cada lado de la igualdad y despejamos la diferencia entre la serie arm\'onica y el logaritmo natural
\begin{align*}
\sum_{k = 1} ^ n \frac{1}{k} - \ln\left(n + \frac{1}{2}\right) &= 1 - \int_n^{n + \frac{1}{2}}\frac{dt}{t} - \int_1^n \frac{t - \floor{t}}{t^2} dt.
\end{align*}
Restamos $\gamma$ a ambos lados de la igualdad
\begin{align*}
\sum_{k = 1} ^ n \frac{1}{k} - \ln\left(n + \frac{1}{2}\right) - \gamma &= 1 - \int_n^{n + \frac{1}{2}}\frac{dt}{t} - \int_1^n \frac{t - \floor{t}}{t^2} dt - \gamma\\
&= \int_n^\infty \frac{t - \floor{t}}{t^2} dt - \int_n^{n + \frac{1}{2}}\frac{dt}{t}
\end{align*}
\end{document}