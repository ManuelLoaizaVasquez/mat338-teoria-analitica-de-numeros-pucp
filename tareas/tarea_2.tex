\documentclass{article}
\usepackage[utf8]{inputenc}
\usepackage{amsfonts,latexsym,amsthm,amssymb,amsmath,amscd,euscript}
\usepackage{mathtools}
\usepackage{framed}
% Descomentar fullpage cuando se quiera utilizar menos margen horizontal
%\usepackage{fullpage}
\usepackage{hyperref}
    \hypersetup{colorlinks=true,citecolor=blue,urlcolor =black,linkbordercolor={1 0 0}}

\newenvironment{statement}[1]{\smallskip\noindent\color[rgb]{1.00,0.00,0.50} {\bf #1.}}{}
\allowdisplaybreaks[1]

\DeclarePairedDelimiter\ceil{\lceil}{\rceil}
\DeclarePairedDelimiter\floor{\lfloor}{\rfloor}

% Comandos para teoremas, definiciones, ejemplos, lemas, etc. para sus respectivos body types.
\renewcommand*{\proofname}{Prueba}
\renewcommand{\contentsname}{Contenido}

\newtheorem{theorem}{Teorema}
\newtheorem*{proposition}{Proposici\'on}
\newtheorem{lemma}[theorem]{Lema}
\newtheorem{corollary}[theorem]{Corolario}
\newtheorem{conjecture}[theorem]{Conjetura}
\newtheorem*{postulate}{Postulado}
\theoremstyle{definition}
\newtheorem{defn}[theorem]{Definici\'on}
\newtheorem{example}[theorem]{Ejemplo}

\theoremstyle{remark}
\newtheorem*{remark}{Observaci\'on}
\newtheorem*{notation}{Notaci\'on}
\newtheorem*{note}{Nota}

% Define tus comandos para hacer la vida m�s f�cil.
\newcommand{\BR}{\mathbb R}
\newcommand{\BC}{\mathbb C}
\newcommand{\BF}{\mathbb F}
\newcommand{\BQ}{\mathbb Q}
\newcommand{\BZ}{\mathbb Z}
\newcommand{\BN}{\mathbb N}

\title{MAT338 Teor\'ia Anal\'itica de N\'umeros}
\author{Manuel Loaiza Vasquez}
\date{Ciclo 2020-1}

\begin{document}

\maketitle

\vspace*{-0.25in}
\centerline{Pontificia Universidad Cat\'olica del Per\'u}
\centerline{Lima, Per\'u}
\centerline{\href{mailto:manuel.loaiza@pucp.edu.pe}{{\tt manuel.loaiza@pucp.edu.pe}}}
\vspace*{0.15in}

\begin{framed}
Segunda tarea del curso de T\'opicos de An\'alisis de la Especialidad de Matem\'aticas dictado en la Facultad de Ciencias e Ingenier\'ia en la Pontificia Universidad Cat\'olica del Per\'u (PUCP) por Alfredo Poirier Schmitz en el ciclo 2020-1.
\end{framed}

\begin{statement}{1}
Conseguir estimados de orden $\cfrac{1}{n^2}$ por arriba y por abajo para
$$\sum_{k \leq n} \frac{1}{k} - \ln\left(n + \frac{1}{2}\right) - \gamma.$$
\end{statement}

Primero hallemos el l\'imite
$$\lim_{n \to \infty} \left(\sum_{k = 1} ^ n \frac{1}{k} - \ln(n + r) \right)$$
para valores positivos de $r$.
\begin{align*}
\sum_{k = 1} ^ n \frac{1}{k} - \ln(n + r) &= \sum_{k = 1} ^ n \frac{1}{k} -\ln(n) + \ln(n) - \ln(n + r)\\
&= \sum_{k = 1} ^ n \frac{1}{k} - \ln(n) + \ln\left(\frac{n}{n + r}\right)\\
&= \sum_{k = 1} ^ n \frac{1}{k} - \ln(n) - \ln\left(\frac{n + r}{n}\right)\\
&= \sum_{k = 1} ^ n \frac{1}{k} - \ln(n) - \ln\left(1 + \frac{r}{n}\right).\\
\end{align*}

Aplicamos aritm\'etica de l\'imites al resultado anterior
\begin{align*}
\lim_{n \to \infty} \left(\sum_{k = 1} ^ n \frac{1}{k} - \ln(n + r)\right) &=
\lim_{n \to \infty} \left(\sum_{k = 1} ^ n \frac{1}{k} - \ln(n)\right) - \lim_{n \to \infty} \left(\ln\left(1 + \frac{r}{n}\right)\right)\\
&= \gamma - 0\\
&= \gamma.
\end{align*}

Asimismo, del corolario $21.2$ tenemos lo la siguientes igualdad
\begin{align*}
\sum_{k = 2} ^ n \frac{1}{k} &= \int_1^n \frac{dt}{t} + \int_1^n (t - \floor{t})\left(-\frac{1}{t^2}\right) dt\\
&= \int_1^n \frac{dt}{t} - \int_1^n \frac{t - \floor{t}}{t^2} dt\\
&= \ln(n) - \int_1^n \frac{t - \floor{t}}{t^2} dt.
\end{align*}

Sumamos uno a cada lado de la igualdad y obtenemos el siguiente resultado
\begin{align*}
1 + \sum_{k = 2} ^ n \frac{1}{k} &= 1 + \ln(n) - \int_1^n \frac{t - \floor{t}}{t^2} dt\\
\sum_{k = 1} ^ n \frac{1}{k} &= 1 + \ln(n) - \int_1^n \frac{t - \floor{t}}{t^2} dt.
\end{align*}

Despejamos la diferencia entre la serie arm\'onica y el logaritmo natural para aplicar l\'imites
\begin{align*}
\sum_{k = 1} ^ n \frac{1}{k} - \ln(n) &= 1 - \int_1^n \frac{t - \floor{t}}{t^2} dt\\
\lim_{n \to \infty} \left(\sum_{k = 1} ^ n \frac{1}{k} - \ln(n)\right) &= 1 - \int_1^\infty \frac{t - \floor{t}}{t^2} dt.
\end{align*}

As\'i obtenemos que la constante de Euler es igual a lo calculado previamente
$$\gamma = 1 - \int_1^\infty \frac{t - \floor{t}}{t^2} dt.$$

Combinaremos los resultados anteriores utilizando $\ln(n + r)$ con $r = \cfrac{1}{2}$
\begin{align*}
\sum_{k = 2} ^ n \frac{1}{k} &= \int_1^n \frac{dt}{t} + \int_1^n (t - \floor{t})\left(-\frac{1}{t^2}\right) dt\\
&= \int_1^n \frac{dt}{t} + \int_n^{n + \frac{1}{2}} \frac{dt}{t} - \int_n^{n + \frac{1}{2}}\frac{dt}{t} - \int_1^n \frac{t - \floor{t}}{t^2} dt\\
&= \int_1^{n + \frac{1}{2}} \frac{dt}{t} - \int_n^{n + \frac{1}{2}}\frac{dt}{t} + - \int_1^n \frac{t - \floor{t}}{t^2} dt\\
&= \ln\left(n + \frac{1}{2}\right) - \int_n^{n + \frac{1}{2}}\frac{dt}{t} - \int_1^n \frac{t - \floor{t}}{t^2} dt.
\end{align*}

Sumamos uno a cada lado de la igualdad y despejamos la diferencia entre la serie arm\'onica y el logaritmo natural
\begin{align*}
\sum_{k = 1} ^ n \frac{1}{k} - \ln\left(n + \frac{1}{2}\right) &= 1 - \int_n^{n + \frac{1}{2}}\frac{dt}{t} - \int_1^n \frac{t - \floor{t}}{t^2} dt.
\end{align*}

Restamos $\gamma$ a ambos lados de la igualdad
\begin{align*}
\sum_{k = 1} ^ n \frac{1}{k} - \ln\left(n + \frac{1}{2}\right) - \gamma &= 1 - \int_n^{n + \frac{1}{2}}\frac{dt}{t} - \int_1^n \frac{t - \floor{t}}{t^2} dt - \gamma\\
&= \int_n^\infty \frac{t - \floor{t}}{t^2} dt - \int_n^{n + \frac{1}{2}}\frac{dt}{t}.
\end{align*}

Ahora nos toca realizar la jugada m\'as astuta del problema, encontraremos una recursi\'on adecuada para obtener una antiderivada f\'acil de aplicarle una desigualdad similar a la no desfasada pero con menor orden
\begin{align*}
\sum_{k = 1} ^ n \frac{1}{k} - \ln\left(n + \frac{1}{2}\right) - \gamma &= \int_n^\infty \frac{t - \floor{t}}{t^2} dt - \int_n^{n + \frac{1}{2}}\frac{dt}{t}\\
&= \int_n^{n + 1} \frac{t - \floor{t}}{t^2}dt + \int_{n + 1}^\infty \frac{t - \floor{t}}{t^2}dt - \int_n^{n + \frac{1}{2}}\frac{dt}{t}\\
&= \int_n^{n + 1} \frac{t}{t^2}dt - \int_n^{n + 1} \frac{\floor{t}}{t^2}dt + \int_{n + 1}^\infty \frac{t - \floor{t}}{t^2}dt - \int_n^{n + \frac{1}{2}}\frac{dt}{t}\\
&= \int_n^{n + 1} \frac{dt}{t}dt - \int_n^{n + \frac{1}{2}}\frac{dt}{t} - \int_n^{n + 1} \frac{\floor{t}}{t^2}dt + \int_{n + 1}^\infty \frac{t - \floor{t}}{t^2}dt\\
&= \int_{n + \frac{1}{2}}^{n + 1}\frac{dt}{t} - \int_n^{n + 1} \frac{\floor{t}}{t^2}dt + \int_{n + 1}^\infty \frac{t - \floor{t}}{t^2}dt\\
&= \int_{n + \frac{1}{2}}^{n + 1}\frac{dt}{t} - \int_n^{n + 1} \frac{\floor{t}}{t^2}dt + \int_{n + 1}^\infty \frac{t - \floor{t}}{t^2}dt + \int_{n + 1}^{n + \frac{3}{2}}\frac{dt}{t} - \int_{n + 1}^{n + \frac{3}{2}}\frac{dt}{t}\\
&= \int_{n + \frac{1}{2}}^{n + 1}\frac{dt}{t} + \int_{n + 1}^{n + \frac{3}{2}}\frac{dt}{t} - \int_n^{n + 1} \frac{\floor{t}}{t^2}dt + \int_{n + 1}^\infty \frac{t - \floor{t}}{t^2}dt - \int_{n + 1}^{n + \frac{3}{2}}\frac{dt}{t}\\
&= \int_{n + \frac{1}{2}}^{n + \frac{3}{2}}\frac{dt}{t} - \int_n^{n + 1} \frac{\floor{t}}{t^2}dt + \int_{n + 1}^\infty \frac{t - \floor{t}}{t^2}dt - \int_{n + 1}^{n + \frac{3}{2}}\frac{dt}{t}\\
&= \int_{n + \frac{1}{2}}^{n + \frac{3}{2}}\frac{dt}{t} - \int_n^{n + 1} \frac{\floor{t}}{t^2}dt + \sum_{k = 1} ^ {n + 1} \frac{1}{k} - \ln\left(n + \frac{3}{2}\right) - \gamma.
\end{align*}

Resolvemos la primera integral
\begin{align*}
\int_{n + \frac{1}{2}}^{n + \frac{3}{2}} \frac{dt}{t} = \ln\left(n + \frac{3}{2}\right) - \ln\left(n + \frac{1}{2}\right).
\end{align*}

Resolvemos la segunda integral
\begin{align*}
\int_n^{n + 1} \frac{\floor{t}}{t^2}dt &= \int_n^{n + 1} \frac{n}{t^2}dt\\
&= n \int_n^{n + 1} \frac{dt}{t^2}\\
&= n \left(-\frac{1}{n + 1} + \frac{1}{n}\right)\\
&= n \left(\frac{1}{n (n + 1)}\right)\\
&= \frac{1}{n + 1}.
\end{align*}

Al reemplazar estos resultados conseguimos la igualdad
$$
\sum_{k = 1} ^ n \frac{1}{k} - \ln\left(n + \frac{1}{2}\right) - \gamma = \ln\left(n + \frac{3}{2}\right) - \ln\left(n + \frac{1}{2}\right) - \frac{1}{n + 1} + \sum_{k = 1} ^ {n + 1} \frac{1}{k} - \ln\left(n + \frac{3}{2}\right) - \gamma.
$$

De este modo, la diferencia entre dos aproximaciones resulta
$$
\left(\sum_{k = 1} ^ n \frac{1}{k} - \ln\left(n + \frac{1}{2}\right) - \gamma\right)
- \left(\sum_{k = 1} ^ {n + 1} \frac{1}{k} - \ln\left(n + \frac{3}{2}\right) - \gamma\right)
= \ln\left(n + \frac{3}{2}\right) - \ln\left(n + \frac{1}{2}\right) - \frac{1}{n + 1}.
$$

Definamos $f: \BR^+ \to \BR$ con
$$f(x) = \ln\left(x + \frac{3}{2}\right) - \ln\left(x + \frac{1}{2}\right) - \frac{1}{x + 1}$$
de donde se se pasa de inmediato a
$$
\left(\sum_{k = 1} ^ n \frac{1}{k} - \ln\left(n + \frac{1}{2}\right) - \gamma\right)
- \left(\sum_{k = 1} ^ {n + 1} \frac{1}{k} - \ln\left(n + \frac{3}{2}\right) - \gamma\right)
= f(n).
$$

Para conseguir una desigualdad telesc\'opica recurrimos a
\begin{align*}
\left(\sum_{k = 1} ^ n \frac{1}{k} - \ln\left(n + \frac{1}{2}\right) - \gamma\right)
- \left(\sum_{k = 1} ^ {n + 1} \frac{1}{k} - \ln\left(n + \frac{3}{2}\right) - \gamma\right)
&= f(n)\\
\left(\sum_{k = 1} ^ {n + 1} \frac{1}{k} - \ln\left(n + \frac{3}{2}\right) - \gamma\right)
- \left(\sum_{k = 1} ^ {n + 2} \frac{1}{k} - \ln\left(n + \frac{5}{2}\right) - \gamma\right)
&= f(n + 1)\\
\vdots\\
\left(\sum_{k = 1} ^ {m} \frac{1}{k} - \ln\left(m + \frac{1}{2}\right) - \gamma\right)
- \left(\sum_{k = 1} ^ {m + 1} \frac{1}{k} - \ln\left(m + \frac{3}{2}\right) - \gamma\right)
&= f(m)
\end{align*}
y sumamos estos t\'erminos consiguiendo
$$
\left(\sum_{k = 1} ^ n \frac{1}{k} - \ln\left(n + \frac{1}{2}\right) - \gamma\right)
- \left(\sum_{k = 1} ^ {m + 1} \frac{1}{k} - \ln\left((m + 1) + \frac{1}{2}\right) - \gamma\right)
= \sum_{i = n}^m f(i).
$$

Aplicamos l\'imites a ambos lados de la igualdad
\begin{align*}
\lim_{m \to \infty} \left[\left(\sum_{k = 1} ^ n \frac{1}{k} - \ln\left(n + \frac{1}{2}\right) - \gamma\right)
- \left(\sum_{k = 1} ^ {m + 1} \frac{1}{k} - \ln\left((m + 1) + \frac{1}{2}\right) - \gamma\right)\right]
&= \lim_{m \to \infty} \sum_{i = n}^m f(i)\\
\left(\sum_{k = 1} ^ n \frac{1}{k} - \ln\left(n + \frac{1}{2}\right) - \gamma\right)
- \lim_{m \to \infty} \left[\left(\sum_{k = 1} ^ {m + 1} \frac{1}{k} - \ln\left((m + 1) + \frac{1}{2}\right) - \gamma\right)\right]
&= \lim_{m \to \infty} \sum_{i = n}^m f(i)\\
\left(\sum_{k = 1} ^ n \frac{1}{k} - \ln\left(n + \frac{1}{2}\right) - \gamma\right)
- (\gamma - \gamma)
&= \lim_{m \to \infty} \sum_{i = n}^m f(i)\\
\sum_{k = 1} ^ n \frac{1}{k} - \ln\left(n + \frac{1}{2}\right) - \gamma &= \sum_{i = n}^\infty f(i).
\end{align*}

Con la idea de conseguir una integral, analicemos la derivada de $f$
\begin{align*}
f'(x) &= \frac{1}{x + \frac{3}{2}} - \frac{1}{x + \frac{1}{2}} + \frac{1}{(x + 1)^2}\\
&= \frac{(x + \frac{1}{2})(x + 1)^2 - (x + \frac{3}{2})(x + 1)^2 + (x + \frac{3}{2})(x + \frac{1}{2})}{(x + \frac{3}{2})(x + 1)^2(x + \frac{1}{2})}\\
&= \frac{-(x + 1)^2 + (x + \frac{3}{2})(x + \frac{1}{2})}{(x + \frac{3}{2})(x + 1)^2(x + \frac{1}{2})}\\
&= \frac{- x^2 - 2x - 1 + x^2 + 2x + \frac{3}{4}}{(x + \frac{3}{2})(x + 1)^2(x + \frac{1}{2})}\\
&= -\frac{1}{4(x + \frac{3}{2})(x + 1)^2(x + \frac{1}{2})}.
\end{align*}

Asimismo, sabemos
\begin{gather*}
x + \frac{3}{2} > x + 1 > x + \frac{1}{2}\\
\frac{1}{x + \frac{3}{2}} < \frac{1}{x + 1} < \frac{1}{x + \frac{1}{2}}
\end{gather*}
y por lo tanto tenemos
$$f'(x) = -\frac{1}{4(x + \frac{3}{2})(x + 1)^2(x + \frac{1}{2})} > -\frac{1}{4(x + \frac{1}{2})^4}.$$

Primero calculemos el l\'imite
\begin{align*}
\lim_{x \to \infty} f(x) &= \lim_{x \to \infty} \left[\ln\left(x + \frac{3}{2}\right) - \ln\left(x + \frac{1}{2}\right) - \frac{1}{x + 1}\right]\\
&= \lim_{x \to \infty} \left[\ln\left(\frac{x + \frac{3}{2}}{x + \frac{1}{2}}\right) - \frac{1}{x + 1}\right]\\
&= \ln(1) - 0\\
&= 0.
\end{align*}

Luego integremos la desigualdad
\begin{align*}
\int_n^{\infty} f'(x) dx &>  -\frac{1}{4} \int_{n}^{\infty} \frac{1}{(x + \frac{1}{2})^4} dx\\
\lim_{m \to \infty} f(x) \Big|_n^m &> -\frac{1}{4} \lim_{m \to \infty} \frac{1}{-3(x + \frac{1}{2})^3} \Big|_n ^ m\\
\lim_{m \to \infty} f(m) -f(n) &> -\frac{1}{12(n + \frac{1}{2})^3}\\
f(n) &< \frac{1}{12(n + \frac{1}{2})^3}.
\end{align*}

Para dos n\'umeros no negativos, sabemos que la media aritm\'etica es mayor o igual a la media geom\'etrica, as\'i que utilizaremos dos n\'umeros consecutivos para que se cancelen al sumar telesc\'opicamente
\begin{align*}
\frac{(n) + (n + 1)}{2} &\geq \sqrt{n(n + 1)}\\
\left(n + \frac{1}{2}\right)^2 &\geq n(n + 1)\\
\left(n + \frac{1}{2}\right)^4 &\geq n^2(n + 1)^2\\
\left(n + \frac{1}{2}\right)\left(n + \frac{1}{2}\right)^3 &\geq n^2(n + 1)^2\\
\frac{1}{\left(n + \frac{1}{2}\right)^3} &\leq \frac{\left(n + \frac{1}{2}\right)}{n^2(n + 1)^2}\\
\frac{1}{\left(n + \frac{1}{2}\right)^3} &\leq \frac{2n + 1}{2n^2(n + 1)^2}\\
\frac{1}{12\left(n + \frac{1}{2}\right)^3} &\leq \frac{2n + 1}{24n^2(n + 1)^2}\\
\frac{1}{12\left(n + \frac{1}{2}\right)^3} &\leq \frac{1}{24}\left[\frac{1}{n^2} - \frac{1}{(n + 1)^2}\right].\\
\end{align*}

Combinamos las dos \'ultimas desigualdades obtenidas
\begin{align*}
f(n) &< \frac{1}{24}\left[\frac{1}{n^2} - \frac{1}{(n + 1)^2}\right]\\
\sum_{i = n}^m f(i) &< \sum_{i = n} ^ m \frac{1}{24}\left[\frac{1}{i^2} - \frac{1}{(i + 1)^2}\right]\\
\sum_{i = n}^m f(i) &< \frac{1}{24}\left[\frac{1}{n^2} - \frac{1}{(m + 1)^2}\right]\\
\lim_{m \to \infty} \sum_{i = n}^m f(i) &< \lim_{m \to \infty} \frac{1}{24}\left[\frac{1}{n^2} - \frac{1}{(m + 1)^2}\right]\\
\sum_{i = n}^\infty f(i) &< \frac{1}{24n^2}.
\end{align*}

Finalmente, reemplazamos esto en la igualdad inicial para obtener el estimado por arriba
$$\sum_{k \leq n} \frac{1}{k} - \ln\left(n + \frac{1}{2}\right) - \gamma = \sum_{i = n}^\infty f(i) < \frac{1}{24n^2}.$$

Para obtener un estimado por abajo, hallaremos otra desigualdad cuando analizamos la derivada. Primero obtegamos la desigualdad
\begin{align*}
\left(x + \frac{3}{2}\right)\left(x + \frac{1}{2}\right) &= x^2 + 2x + \frac{3}{4}\\
&= (x + 1)^2 - \frac{1}{4}\\
&< (x + 1) ^ 2\\
(x + 1)^2\left(x + \frac{3}{2}\right)\left(x + \frac{1}{2}\right) &< (x + 1)^4\\
-4(x + 1)^2\left(x + \frac{3}{2}\right)\left(x + \frac{1}{2}\right) &> -4(x + 1)^4\\
-\frac{1}{4(x + 1)^2\left(x + \frac{3}{2}\right)\left(x + \frac{1}{2}\right)} &< -\frac{1}{4(x + 1)^4}\\
f'(x) &< -\frac{1}{4(x + 1)^4}.
\end{align*}

Integramos esta desigualdad y sumamos los t\'erminos similar a la suma telesc\'opica anterior, solo que en este caso utilizaremos el criterio de comparaci\'on de la serie con la integral
\begin{align*}
\int_n^\infty f'(x) &< -\frac{1}{4}\int_n^\infty \frac{1}{(x + 1)^4}dx\\
\lim_{m \to \infty} f(x) \Big|_n^m &< -\frac{1}{4} \lim_{m \to \infty} \frac{1}{-3(x + 1)^3} \Big|_n ^ m\\
\lim_{m \to \infty} f(m) -f(n) &< -\frac{1}{12(n + 1)^3}\\
f(n) &> \frac{1}{12(n + 1)^3}\\
\sum_{i = n}^m f(i) &> \sum_{i = n}^m \frac{1}{12(n + 1) ^ 3}\\
\lim_{m \to \infty}\sum_{i = n}^m f(i) &> \lim_{m \to \infty}\sum_{i = n}^m \frac{1}{12(n + 1) ^ 3}\\
\sum_{i = n}^\infty f(i) &> \lim_{m \to \infty}\int_n^m \frac{1}{12(x + 1) ^ 3}dx\\
\sum_{i = n}^\infty f(i) &> \frac{1}{24(n + 1)^2}\\
\sum_{k \leq n} \frac{1}{k} - \ln\left(n + \frac{1}{2}\right) - \gamma &> \frac{1}{24(n + 1)^2}.
\end{align*}

Por \'ultimo, queda establecido el estimado
$$
\frac{1}{24(n + 1)^2}
< \left(\sum_{k \leq n} \frac{1}{k} - \ln\left(n + \frac{1}{2}\right)\right) - \gamma
< \frac{1}{24n^2}.
$$
\end{document}