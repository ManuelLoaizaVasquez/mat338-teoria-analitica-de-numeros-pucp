\documentclass{article}
\usepackage[utf8]{inputenc}
\usepackage{amsfonts,latexsym,amsthm,amssymb,amsmath,amscd,euscript}
\usepackage{mathtools}
\usepackage{framed}
% Descomentar fullpage cuando se quiera utilizar menos margen horizontal
%\usepackage{fullpage}
\usepackage{hyperref}
    \hypersetup{colorlinks=true,citecolor=blue,urlcolor =black,linkbordercolor={1 0 0}}
\usepackage{listings}
\usepackage[usenames,dvipsnames]{color}

% Configuracion de Listings
\definecolor{darkblue}{rgb}{0,0,0.4}
\lstloadlanguages{C++}
\lstnewenvironment{code}
	{%\lstset{	numbers=none, frame=lines, basicstyle=\small\ttfamily, }%
	 \csname lst@SetFirstLabel\endcsname}
	{\csname lst@SaveFirstLabel\endcsname}
\lstset{% general command to set parameter(s)
	language=C++, basicstyle=\small\ttfamily, keywordstyle=\slshape,
	emph=[1]{tipo,usa}, emphstyle={[1]\sffamily\bfseries},
	morekeywords={tint,forn,forsn,fore},
	basewidth={0.47em,0.40em},
	columns=fixed, fontadjust, resetmargins, xrightmargin=5pt, xleftmargin=15pt,
	flexiblecolumns=false, tabsize=2, breaklines,	breakatwhitespace=false, extendedchars=true,
	numbers=left, numberstyle=\tiny, stepnumber=1, numbersep=9pt,
	frame=l, framesep=3pt,
    basicstyle=\ttfamily,
    keywordstyle=\color{darkblue}\ttfamily,
    stringstyle=\color{magenta}\ttfamily,
    commentstyle=\color{RedOrange}\ttfamily,
    morecomment=[l][\color{OliveGreen}]{\#}
}

\lstdefinestyle{C++}{
	language=C++, basicstyle=\small\ttfamily, keywordstyle=\slshape,
	emph=[1]{tipo,usa,tipo2}, emphstyle={[1]\sffamily\bfseries},
	morekeywords={tint,forn,forsn,fore},
	basewidth={0.47em,0.40em},
	columns=fixed, fontadjust, resetmargins, xrightmargin=5pt, xleftmargin=15pt,
	flexiblecolumns=false, tabsize=2, breaklines,	breakatwhitespace=false, extendedchars=true,
	numbers=left, numberstyle=\tiny, stepnumber=1, numbersep=9pt,
	frame=l, framesep=3pt,
    basicstyle=\ttfamily,
    keywordstyle=\color{darkblue}\ttfamily,
    stringstyle=\color{magenta}\ttfamily,
    commentstyle=\color{RedOrange}\ttfamily,
    morecomment=[l][\color{OliveGreen}]{\#}
}
\newcommand\cppfile[2][]{\lstinputlisting[style=C++,linerange={#1}]{#2}}

% Enunciado de los problemas
\newenvironment{statement}[1]{\smallskip\noindent\color[rgb]{1.00,0.00,0.50} {\bf #1.}}{}
\allowdisplaybreaks[1]

\DeclarePairedDelimiter\ceil{\lceil}{\rceil}
\DeclarePairedDelimiter\floor{\lfloor}{\rfloor}

% Comandos para teoremas, definiciones, ejemplos, lemas, etc. para sus respectivos body types.
\renewcommand*{\proofname}{Prueba}
\renewcommand{\contentsname}{Contenido}

\newtheorem{theorem}{Teorema}
\newtheorem*{proposition}{Proposici\'on}
\newtheorem{lemma}[theorem]{Lema}
\newtheorem{corollary}[theorem]{Corolario}
\newtheorem{conjecture}[theorem]{Conjetura}
\newtheorem*{postulate}{Postulado}
\theoremstyle{definition}
\newtheorem{defn}[theorem]{Definici\'on}
\newtheorem{example}[theorem]{Ejemplo}

\theoremstyle{remark}
\newtheorem*{remark}{Observaci\'on}
\newtheorem*{notation}{Notaci\'on}
\newtheorem*{note}{Nota}

% Define tus comandos para hacer la vida m�s f�cil.
\newcommand{\BR}{\mathbb R}
\newcommand{\BC}{\mathbb C}
\newcommand{\BF}{\mathbb F}
\newcommand{\BQ}{\mathbb Q}
\newcommand{\BZ}{\mathbb Z}
\newcommand{\BN}{\mathbb N}

\title{MAT338 Teor\'ia Anal\'itica de N\'umeros}
\author{Manuel Loaiza Vasquez}
\date{Ciclo 2020-1}

\begin{document}

\maketitle

\vspace*{-0.25in}
\centerline{Pontificia Universidad Cat\'olica del Per\'u}
\centerline{Lima, Per\'u}
\centerline{\href{mailto:manuel.loaiza@pucp.edu.pe}{{\tt manuel.loaiza@pucp.edu.pe}}}
\vspace*{0.15in}

\begin{framed}
\'Ultima tarea del curso de T\'opicos de An\'alisis de la Especialidad de Matem\'aticas dictado en la Facultad de Ciencias e Ingenier\'ia en la Pontificia Universidad Cat\'olica del Per\'u (PUCP) por Alfredo Poirier Schmitz en el ciclo 2020-1.
\end{framed}

\begin{statement}{1}
Para todo n\'umero real $x$ mayor o igual a $1$ se cumple la f\'ormula de Selberg
$$\sum_{p \leq x} \ln^2(p) + \sum_{pq \leq x} \ln(p) \ln(q) = 2x\ln(x) + O(x).$$
\end{statement}
 
\begin{lemma}
Para todo $x \geq 1$ tenemos
$$\sum_{n \leq x} \frac{1}{n} = \ln x + \gamma + O\left(\frac{1}{x}\right).$$
\end{lemma}

\begin{proof}
Sea $f : [1, \infty) \to \BR$ con $f(x) = \frac{1}{x}$, sabemos que $f$ es continua y diferenciable en todo su dominio,
en particular, en el intervalo $[1, x]$; por lo tanto, podemos aplicar la f\'ormula de sumaci\'on de Euler.

Primero, utilizar\'e la f\'ormula de sumaci\'on de Euler en el intervalo $[2, k]$, con $k$ un n\'umero entero mayor o igual a 2
\begin{align*}
\sum_{n = 2}^k \frac{1}{n} &= \int_1^k \frac{dt}{t} + \int_{1}^k (t - \floor{t})\left(\frac{1}{-t^2}\right) dt \\
&= (\ln t) \big|_1^k - \int_1^k \frac{t - \floor{t}}{t^2} dt \\
1 + \sum_{n = 2}^k \frac{1}{n} &= 1 + \ln k - \int_1^k \frac{t - \floor{t}}{t^2} dt \\
\sum_{n = 1}^k \frac{1}{n} &= \ln k + 1 - \int_1^k \frac{t - \floor{t}}{t^2} dt \\
\sum_{n = 1}^k \frac{1}{n} - \ln k &= 1 - \int_1^k \frac{t - \floor{t}}{t^2} dt.
\end{align*}

Analicemos que pasa cuando $k \to \infty$
$$\gamma = \lim_{k \to \infty} \left(\sum_{n = 1}^k \frac{1}{n} - \ln k\right) 
= \lim_{k \to \infty} \left(1 - \int_1^k \frac{t - \floor{t}}{t^2}\right) dt.$$
Este l\'imite existe, puesto que
\begin{align*}
\lim_{k \to \infty} \int_1^k \frac{t - \floor{t}}{t^2} dt &\leq \lim_{k \to \infty} \int_1^k \frac{1}{t^2} dt \\
&= \lim_{k \to \infty} \left(-\frac{1}{t}\right)\Big|_1^k dt \\
&= 1
\end{align*}
converge y por aritm\'etica de l\'imites, $\gamma = \lim_{k \to \infty} \left(1 - \int_1^k \frac{t - \floor{t}}{t^2}\right) dt$ converge.

Finalmente, utilizar\'e la f\'ormula de sumaci\'on de Euler para $x$ un n\'umero real mayor o igual a 1
\begin{align*}
\sum_{n \leq x} \frac{1}{n} &= \int_1^x \frac{1}{t} dt + \int_1^x (t - \floor{t})\left(\frac{1}{-t^2}\right) dt 
+ (1 - \floor{1}) - (x - \floor{x})\left(\frac{1}{x}\right) \\
&= (\ln t) \big|_1^x - \int_1^x \frac{t - \floor{t}}{t^2} dt - \frac{x - \floor{x}}{x} \\
&= \ln x - \int_1^x \frac{t - \floor{t}}{t^2} dt - \frac{x - \floor{x}}{x} \\
&\leq \ln x - \int_1^x \frac{t - \floor{t}}{t^2} dt + 1 + \frac{x - \floor{x}}{x} \\
\sum_{n \leq x} \frac{1}{n} &= \ln x - \int_1^x \frac{t - \floor{t}}{t^2} dt + 1 + O\left(\frac{1}{x}\right) \\
&= \ln x - \int_1^x \frac{t - \floor{t}}{t^2} dt - \int_x^\infty \frac{t - \floor{t}}{t^2} dt 
+ \int_x^\infty \frac{t - \floor{t}}{t^2} dt + 1 + O\left(\frac{1}{x}\right) \\
&= \ln x - \int_1^\infty \frac{t - \floor{t}}{t^2} dt + \int_x^\infty \frac{t - \floor{t}}{t^2} dt + 1 + O\left(\frac{1}{x}\right) \\
&= \ln x + \left(1 - \int_1^\infty \frac{t - \floor{t}}{t^2} dt\right) + \int_x^\infty \frac{t - \floor{t}}{t^2} dt + O\left(\frac{1}{x}\right) \\
&= \ln x + \gamma + \int_x^\infty \frac{t - \floor{t}}{t^2} dt + O\left(\frac{1}{x}\right) \\
&\leq \ln x + \gamma + \int_x^\infty \frac{dt}{t^2} + O\left(\frac{1}{x}\right) \\
&= \ln x + \gamma + \lim_{b \to \infty} \left(-\frac{1}{t}\right) \Big|_x^b + O\left(\frac{1}{x}\right) \\
&= \ln x + \gamma + \frac{1}{x} + O\left(\frac{1}{x}\right) \\
&= \ln x + \gamma + O\left(\frac{1}{x}\right)
\end{align*}
\end{proof}

\begin{lemma}
Para todo $x \geq 1$ tenemos
$$\sum_{n \leq x} \ln n = x\ln x - x + O(\ln x).$$
\end{lemma}

\begin{proof}
Sea $f : [1, \infty) \to \BR$ con $f(x) = \ln x$, sabemos que $f$ es continua y diferenciable en todo su dominio, 
en particular, en el intervalo $[1, x]$; por lo tanto, podemos aplicar la f\'ormula de sumaci\'on de Euler
\begin{align*}
\sum_{n \leq x} \ln x &= \int_1^x \ln t dt + \int_1^x (t - \floor{t})\left(\frac{1}{t}\right) dt + (\floor{x} - x) \ln x - (\floor{1} - 1) \ln 1 \\
&= (t \ln t - t) \big|_1^x + \int_1^x \frac{(t - \floor{t})}{t} dt + (\floor{x} - x) \ln x \\
&= x \ln x - x + 1 + \int_1^x \frac{t - \floor{t}}{t} dt + (\floor{x} - x) \ln x \\
&\leq x \ln x - x + 1 + \int_1^x \frac{1}{t} dt + \ln x \\
&= x \ln x - x + 1 + 2 \ln x.
\end{align*}

Concluimos que $\sum_{n \leq x} \ln n = x\ln x - x + O(\ln x)$.
\end{proof}

\begin{lemma}
Para toda funci\'on aritm\'etica $f$ se cumple
$$\sum_{n \leq x} \sum_{d \mid n} f(d) = \sum_{n \leq x} f(n) \floor{\frac{x}{n}}.$$
\end{lemma}

\begin{proof}
Sea $f$ y $g$ dos funciones aritm\'eticas, $F$ y $G$ sus respectivas cumulativas;
es decir, $F(x) = \sum_{n \leq x} f(x)$ y $G(x) = \sum_{n \leq x} g(x)$,
analicemos la cumulativa del producto de Dirichlet de $f$ y $g$
\begin{align*}
\sum_{n \leq x} f * g (n) &= \sum_{n \leq x} \sum_{cd = n} f(c)g(d) \\
&= \sum_{c \leq x} \sum_{d \leq \frac{x}{c}} f(c) g(d) \\
&= \sum_{c \leq x} f(c) \sum_{d \leq \frac{x}{c}} g(d) \\
&= \sum_{c \leq x} f(c) G\left(\frac{x}{c}\right).
\end{align*}

En particular, cuando $g = 1$, su cumulativa es
$$\sum_{n \leq x} 1(n) = \sum_{n \leq x} 1 = \floor{x}$$
y aplicamos lo anterior directamente expres\'ando lo que queremos probar como un producto de Dirichlet
$$\sum_{n \leq x} \sum_{d \mid n} f(d) = \sum_{n \leq x} f * 1 (n) = \sum_{n \leq x} f(n) \floor{\frac{x}{n}}$$
concluyendo con la demostraci\'on.
\end{proof}

\begin{lemma}
Para todo n\'umero real $x$ tenemos
$$\floor{x} = x + O(1).$$
\end{lemma}

\begin{proof}
Sea $x = n + r$ un n\'umero real no negativo, con $n \in \BZ$ y $0 \leq r < 1$. De esta manera, por definici\'on de m\'aximo entero tenemos
\begin{align*}
\floor{x} &= n \\
&= x - r \\
&= x + O(1).
\end{align*}
Concluimos trivialmente que $\floor{x} = x + O(1)$.
\end{proof}

\begin{lemma}
Para todo $x \geq 1$ tenemos
$$\Psi(x) = O(x).$$
\end{lemma}

\begin{proof}
Utilizar\'e teorema de Chebyshev obtenido en la clase en la cual desarrollamos el paper \textit{Elementary methods in the study of the distribution of prime numbers} de Harold Diamond
$$A \leq \liminf \frac{\Psi(x)}{x} \leq \limsup \frac{\Psi(x)}{x} \leq \frac{6A}{5}$$
con $A = -\frac{\ln 1}{1} + \frac{\ln 2}{2} + \frac{\ln 3}{3} + \frac{\ln 5}{5} - \frac{\ln 30}{30} \approx 0.92129202293409078091340844996160...$
Para analizar el comportamiento asint\'otico nos centraremos en la parte derecha de la desigualdad reescribi\'endola con valor absoluto ya que es una funci\'on positiva
$$\limsup \frac{|\Psi(x)|}{x} \leq \frac{6A}{5}.$$
Por definici\'on, sabemos que existe un $n_0$ a partir del cual
$$\frac{|\Psi(x)|}{x} \leq \frac{6A}{5}$$
para todo $x > n_0$. Pasamos a multiplicar la funci\'on lineal y obtenemos
$$|\Psi(x)| \leq \left(\frac{6A}{5}\right) x$$
con $\frac{6A}{5} > 0$, lo cual denotamos con $\Psi(x) = O(x)$.
\end{proof}

\begin{lemma}
La funci\'on de Mangoldt se puede expresar como el siguiente producto de Dirichlet
$$\Lambda = \mu * \ln.$$
\end{lemma}

\begin{proof}
Sabemos que $\Lambda * 1 = \ln$ y aplicamos los propiedades del producto de Dirichlet
\begin{align*}
(\Lambda * 1) * \mu &= \ln * \mu \\
\Lambda * (1 * \mu) &= \ln * \mu \\
\Lambda * \mathbb{U} &= \ln * \mu \\
\Lambda &= \ln * \mu \\
\Lambda &= \mu * \ln.
\end{align*}
Concluimos que $\Lambda = \mu * \ln$.
\end{proof}

\begin{lemma}
Para todo $x \geq 1$ tenemos
$$\sum_{n \leq x} \frac{\Lambda(n)}{n} = \ln x + O(1).$$
\end{lemma}

\begin{proof}
Sabemos que $\ln = \Lambda * 1$, por lo tanto, desarrollamos
\begin{align*}
\ln n &= \sum_{d \mid n} \Lambda(d) \\
\sum_{n \leq x} \ln n &= \sum_{n \leq x} \sum_{d \mid n} \Lambda(d).
\end{align*}

Del Lema 3 obtenemos
\begin{align*}
\sum_{n \leq x} \ln n &= \sum_{n \leq x} \Lambda(n) \floor{\frac{x}{n}}
\end{align*}
y utilizando el Lema 4 conseguimos
\begin{align*}
\sum_{n \leq x} \ln n &= \sum_{n \leq x} \Lambda(n) \left(\frac{x}{n} + O(1)\right) \\
&= x \sum_{n \leq x} \frac{\Lambda(n)}{n} + O\left(\sum_{n \leq x} \Lambda(x) \right) \\
&= x \sum_{n \leq x} \frac{\Lambda(n)}{n} + O(\Psi(x)).
\end{align*}

La astucia de Chebyshev es la que permite mejorar el estimado que obtuvimos en clase de 
$\Psi(x) = \vartheta(x) + O(\sqrt{x} \ln^2 x)$ con $\vartheta(x) = O(x \ln x)$ 
por $\Psi(x) = O(x)$ gracias al Lema 5.
Utilizando este poderoso teorema de Chebyshev y aplicando el Lema 2 en la suma acumulada de logaritmos conseguimos
\begin{align*}
x\ln x - x + O(\ln x) &= x \sum_{n \leq x} \frac{\Lambda(n)}{n} + O(x)
\end{align*}
y despejamos
\begin{align*}
\sum_{n \leq x} \frac{\Lambda(n)}{n} &= \frac{x \ln x}{x} - 1 + O\left(\frac{\ln x}{x}\right) + O(1) \\
&= \ln x - 1 + O(1) + O\left(\frac{\ln x}{x}\right) \\
&\leq \ln x - 1 + c_0 + O(1) \\
&\leq \ln x + c_1 \\
&= \ln x + O(1).
\end{align*}

Concluimos que $\sum_{n \leq x} \frac{\Lambda(n)}{n} = \ln x + O(1)$.
\end{proof}

\begin{lemma}
Para todo $f, g : [1, \infty) \to \BR$ con $g(x) = \sum_{n \leq x} f\left(\frac{x}{n}\right) \ln x$ tenemos
$$\sum_{n \leq x} \mu(n) g\left(\frac{x}{n}\right) = f(x) \ln(x) + \sum_{n \leq x} f\left(\frac{x}{n}\right) \Lambda(n).$$
\end{lemma}

\begin{proof}
Desarrollemos la sumatoria que queremos analizar
\begin{align*}
\sum_{n \leq x} \mu(n) g\left(\frac{x}{n}\right) &= \sum_{cd \leq x} \mu(c)g(d) \\
&= \sum_{cd \leq x} \mu(c) \sum_{e \leq d} f\left(\frac{d}{e}\right) \ln(d) \\
&= \sum_{cd \leq x} \mu(c) \ln\left(\frac{x}{c}\right) f \left(\frac{x}{cd}\right) \\
&= \sum_{n \leq x} f\left(\frac{x}{n}\right) \sum_{d \mid n} \mu(d) \ln\left(\frac{x}{n}\right) \\
&= \sum_{n \leq x} f\left(\frac{x}{n}\right) \sum_{d \mid n} \mu(d) \left[\ln \left(\frac{x}{n}\right) + \ln\left(\frac{n}{d}\right)\right] \\
&= \left[\sum_{n \leq x} f\left(\frac{x}{n}\right) \ln\left(\frac{x}{n}\right) \sum_{d \mid n} \mu(d)\right]
+  \left[\sum_{n \leq x} f\left(\frac{x}{n}\right) \sum_{d \mid n} \mu(d) \ln\left(\frac{n}{d}\right)\right].
\end{align*}

Asimismo, sabemos que $\mu * 1 = \mathbb{U}$ y
sabemos que el \'unico valor de $\mathbb{U}$ distinto de cero se obtiene en uno.
Por consiguiente, la igualdad se reduce a
\begin{align*}
\sum_{n \leq x} \mu(n)g\left(\frac{x}{n}\right) &= f(x)\ln x + \sum_{n \leq x} f\left(\frac{x}{n}\right) (\mu * \ln)(n).
\end{align*}

Finalmente, utilizamos el Lema $6$ para concluir que $\sum_{n \leq x} \mu(n)g\left(\frac{x}{n}\right) = f(x)\ln x + \sum_{n \leq x} f\left(\frac{x}{n}\right)\Lambda(n)$.
\end{proof}

\begin{lemma}
Para todo $x \geq 1$ tenemos
$$\ln^2 x = O(\sqrt{x}).$$
\end{lemma}

\begin{proof}
Sabemos que para todo $x \geq 1$ se cumple que $x > \ln x$ analizando simplemente la derivada de $x - \ln x$.
\begin{align*}
\ln^2 x &= \ln^2 ((x ^ {\frac{1}{4}}) ^ 4) \\
&= 16 \ln^2 (x ^ {\frac{1}{4}}) \\
&< 16 (x ^ {\frac{1}{4}}) ^ 2 \\
&= 16 \sqrt{x}.
\end{align*}

Por lo tanto, concluimos que $\ln^2 x = O(\sqrt{x})$.
\end{proof}

\begin{lemma}
Para todo $x \geq 1$ tenemos
$$\Psi(x) \ln x + \sum_{n \leq x} \Psi\left(\frac{x}{n}\right)\Lambda(n) = 2x\ln x + O(x).$$
\end{lemma}

\begin{proof}
Para poder utilizar el Lema 8, debemos definir convenientemente la funci\'on $f : [1, \infty) \to \BR$ como
$$f(x) = \Psi(x) - x + \gamma + 1.$$
y antes de aplicar el Lema 8 le daremos forma a la funci\'on $g(x) = \sum_{n \leq x} f\left(\frac{x}{n}\right)\ln x$
\begin{align*}
\sum_{n \leq x} f\left(\frac{x}{n}\right)\ln x &= \sum_{n \leq x} \left(\Psi\left(\frac{x}{n}\right) - \frac{x}{n} + \gamma + 1\right) \ln x \\
&= \sum_{n \leq x} \Psi\left(\frac{x}{n}\right) \ln x - x \ln x \sum_{n \leq x} \frac{1}{n} + (\gamma + 1) \ln x \sum_{n \leq x} 1.
\end{align*}

Analicemos por separado a cada sumatoria. Primero $\sum_{n \leq x} \Psi\left(\frac{x}{n}\right)$
\begin{align*}
\sum_{n \leq x} \Psi\left(\frac{x}{n}\right) &= \sum_{n \leq x} \sum_{d \leq \frac{x}{n}} \Lambda(d) \\
&= \sum_{n \leq x} \sum_{d \mid n} \Lambda(d) \\
&= \sum_{n \leq x} (\Lambda * 1)(n) \\
&= \sum_{n \leq x} \ln n
\end{align*}
el cual es igual a $x\ln x - x + O(\ln x)$ por el Lema 2 y multiplic\'andole el logaritmo obtenemos
$$\sum_{n \leq x} \Psi\left(\frac{x}{n}\right)\ln x = x \ln^2 x - x \ln x + O(\ln^2 x).$$

Ahora analicemos la segunda sumatoria utilizando el Lema 1
\begin{align*}
-x \ln x \sum_{n \leq x} \frac{1}{n} &= -x \ln x \left(\ln x + \gamma + O\left(\frac{1}{x}\right)\right) \\
&= -x \ln^2 x - \gamma x \ln x + O(\ln x).
\end{align*}

Finalmente, analicemos la tercera sumatorio utilizando el Lema 4
\begin{align*}
(\gamma + 1) \ln x \sum_{n \leq x} 1 &= (\gamma + 1) \ln x \floor{x} \\
&= (\gamma + 1) \ln x (x + O(1)) \\
&= (\gamma + 1) x \ln x + O(\ln x).
\end{align*}

Juntando los tres resultados obtenemos
\begin{align*}
g(x) &= x \ln^2 x - x \ln x + O(\ln^2 x) - x \ln^2 x - \gamma x \ln x + O(\ln x) + (\gamma + 1) x \ln x + O(\ln x) \\
&= O(\ln^2 x) + O(\ln x) \\
&= O(\ln^2 x)
\end{align*}

Utilizando el Lema 8 obtenemos la igualdad
$$\sum_{n \leq x} \mu(n) g\left(\frac{x}{n}\right) = (\Psi(x) - x + \gamma + 1) \ln x + \sum_{n \leq x} \left(\Psi\left(\frac{x}{n}\right) - \frac{x}{n} + \gamma + 1\right) \Lambda(n)$$
y analizaremos ambos lados de la desigualdad por separado.
Utilizando la desigualdad triangular en la secci\'on de la izquierda y luego el hecho de que $g(x) = O(\ln^2 x)$ obtenemos
\begin{align*}
\sum_{n \leq x} \mu(n) g\left(\frac{x}{n}\right) &\leq \sum_{n \leq x} g\left(\frac{x}{n}\right) \\
&= O\left(\sum_{n \leq x} g\left(\frac{x}{n}\right)\right) \\
&= O\left(\sum_{n \leq x} \ln^2 \left(\frac{x}{n}\right)\right).
\end{align*}

Utilizamos el Lema 9 para conseguir
\begin{align*}
\sum_{n \leq x} \mu(n) g\left(\frac{x}{n}\right) &= O\left(\sum_{n \leq x} \sqrt{\frac{x}{n}}\right) \\
&= O\left(\sqrt{x} \sum_{n \leq x} \frac{1}{\sqrt{n}}\right).
\end{align*}

Ahora utilizar\'e otro teorema que probamos en la secci\'on en la cual estudiamos a la funci\'on $\zeta$ de Riemann
$$\sum_{n \leq x} \frac{1}{n^\sigma} = \frac{x^{1 - \sigma}}{1 - \sigma} + O(1) = O(x ^ {1 - \sigma}), \, 0 < \sigma < 1$$
para el caso particular cuando $\sigma = \frac{1}{2}$.
\begin{align*}
\sum_{n \leq x} \mu(n) g\left(\frac{x}{n}\right) &= O(\sqrt{x}\sqrt{x}) \\
&= O(x).
\end{align*}

Ahora desarrollar\'e la parte de la derecha de la igualdad inicial
$$(\Psi(x) - x + \gamma + 1) \ln x + \sum_{n \leq x} \left(\Psi\left(\frac{x}{n}\right) - \frac{x}{n} + \gamma + 1\right) \Lambda(n)$$
\begin{align*}
&= \Psi(x)\ln x + (-x + \gamma + 1) \ln x + \sum_{n \leq x} \Psi\left(\frac{x}{n}\right)\Lambda(n) - x \sum_{n \leq x} \frac{\Lambda(n)}{n} + (\gamma + 1) \sum_{n \leq x} \Lambda(n) \\
&= \Psi(x)\ln x + (-x + \gamma + 1) \ln x + \sum_{n \leq x} \Psi\left(\frac{x}{n}\right)\Lambda(n) - x \sum_{n \leq x} \frac{\Lambda(n)}{n} + (\gamma + 1) \Psi(x) \\
&= \Psi(x)\ln x + \sum_{n \leq x} \Psi\left(\frac{x}{n}\right)\Lambda(n) - x \ln x - x \sum_{n \leq x} \frac{\Lambda(n)}{n} + (\gamma + 1)\Psi(x)
\end{align*}
y utilizando el Lema 5 y el Lema 7 reducimos la expresi\'on
\begin{align*}
&= \Psi(x)\ln x + \sum_{n \leq x} \Psi\left(\frac{x}{n}\right)\Lambda(n) - x \ln x - x(\ln x + O(1)) + O(x) \\
&= \Psi(x)\ln x + \sum_{n \leq x} \Psi\left(\frac{x}{n}\right)\Lambda(n) - 2x \ln x + O(x).
\end{align*}

Finalmente, igualamos los resultados de ambas partes
\begin{align*}
\Psi(x)\ln x + \sum_{n \leq x} \Psi\left(\frac{x}{n}\right)\Lambda(n) - 2x \ln x + O(x) &= O(x)\\
\Psi(x)\ln x + \sum_{n \leq x} \Psi\left(\frac{x}{n}\right)\Lambda(n) &= 2x \ln x + O(x)
\end{align*}
concluyendo con la prueba del lema.
\end{proof}

\begin{lemma}
Para todo $x \geq 1$ tenemos
$$\Psi(x) = \vartheta(x) + O(\sqrt{x}\ln x).$$
\end{lemma}

\begin{proof}
Sabemos que
$$\Psi(x) = \sum_{n = 1}^\infty \vartheta(x^\frac{1}{n})$$
y al mismo tiempo esa sumatoria tiene una cantidad finita de elementos puesto que la funci\'on $\vartheta$ 
solo tiene sentido cuando es evaluada en valores mayores o iguales a $2$. Hallamos ese momento $m$
\begin{align*}
x^\frac{1}{m} &\geq 2 \\
x^\frac{2}{m} &\geq 4 \\
x^\frac{2}{m} &> e \\
\frac{2}{m}\ln x &> \ln e \\
\frac{2}{m}\ln x &> 1 \\
m &< 2 \ln x
\end{align*}
y notamos que para valores mayores $m = \floor{2 \ln x}$ los elementos de la suma son iguales a cero.
Ahora podemos escribir a $\Psi$ de la siguiente forma
\begin{align*}
\Psi(x) &= \vartheta(x) + \vartheta(x^\frac{1}{2}) + \cdots + \vartheta(x^\frac{1}{m}) \\
\Psi(x) &= \vartheta(x) + \sum_{n = 2}^m \vartheta(x^\frac{1}{n}).
\end{align*}

Analicemos a la sumatoria $\sum_{n = 2}^m \vartheta(x^\frac{1}{n})$
$$\sum_{n = 2}^m \vartheta(x^\frac{1}{n}) = \sum_{n = 2}^m \sum_{p \leq x^\frac{1}{n}} \ln p$$
y tratemos de darle forma para manipularla de una manera sencilla, es decir,
cambiar la indexaci\'on para poder extraer el logaritmo en la sumatoria anidada.
Observemos que si un n\'umero primo $p$ ser\'a incluido en la sumatoria, su logaritmo contribuir\'a a la sumatoria
tantas veces como las ra\'ices de $x$ lo permitan
\begin{align*}
p \leq x^\frac{1}{2} &\implies p^2 \leq x \\
p \leq x^\frac{1}{3} &\implies p^3 \leq x \\
&\,\,\,\,\,\,\,\vdots \\
p \leq x^\frac{1}{k} &\implies p^k \leq x.
\end{align*}
Esto quiere decir que quiero hallar el m\'aximo $k$ fijado un primo $p$, despejando obtenemos
\begin{align*}
p^k &\leq x \\
k \ln p &\leq \ln x \\
k &\leq \frac{\ln x}{\ln p} \\
k &= \floor{\frac{\ln x}{\ln p}}.
\end{align*}
Asimismo, notamos que estos primos son menores iguales que la ra\'iz de $x$, el cual es un detalla importante del cual nos aprovecharemos.
Ahora analicemos la forma equivalente de la doble sumatoria
\begin{align*}
\sum_{n = 2}^m \sum_{p \leq x^\frac{1}{n}} \ln p &= \sum_{p \leq \sqrt{x}} \sum_{2 \leq n \leq \floor{\frac{\ln x}{\ln p}}} \ln p \\
&= \sum_{p \leq \sqrt{x}} \ln p \sum_{2 \leq n \leq \floor{\frac{\ln x}{\ln p}}} 1 \\
&\leq \sum_{p \leq \sqrt{x}} \ln p \floor{\frac{\ln x}{\ln p}} \\
&\leq \sum_{p \leq \sqrt{x}} \ln p \left(\frac{\ln x}{\ln p}\right) \\
&= \sum_{p \leq \sqrt{x}} \ln x \\
&= \ln x \sum_{p \leq \sqrt{x}} 1 \\
&\leq \ln x \sum_{n \leq \sqrt{x}} 1 \\
&\leq \ln x \sqrt{x} \\
\sum_{n = 2}^m \vartheta(x^\frac{1}{n}) &= O(\sqrt{x} \ln x).
\end{align*}

Utilizamos esto para concluir que $\Psi(x) = \vartheta(x) + O(\sqrt{x} \ln x)$.
\end{proof}

\begin{lemma}
Sea $p$ un n\'umero primo, la serie
$$\sum_{p = 2}^{\infty} \frac{\ln p}{p (p - 1)}$$
converge.
\end{lemma}

\begin{proof}
Primero hay que analizar el siguiente l\'imite
$$\lim_{p \to \infty} \frac{\ln p}{p (p - 1)} p^\frac{3}{2}$$
y antes de eso reescribiremos lo que queremos analizar de la siguiente manera
$$\frac{\ln p}{p (p - 1)} p^\frac{3}{2} = \left(\frac{\ln p}{\sqrt{p}}\right)\left(\frac{p}{p - 1}\right).$$
Si ambos l\'imites existen, tendr\'iamos que el l\'imite de nuestra expresi\'on es el producto.

Analicemos el l\'imite de lo que est\'a dentro del primer par\'entesis
$$\lim_{p \to \infty} \frac{\ln p}{\sqrt{p}} = \frac{\infty}{\infty},$$
como este es de la forma $\frac{\infty}{\infty}$, aplicamos la regla de L'Hospital
$$\lim_{p \to \infty} \frac{\ln p}{\sqrt{p}}
= \lim_{p \to \infty}\frac{\frac{1}{p}}{\frac{1}{2\sqrt{p}}}
= \lim_{p \to \infty} \frac{2}{\sqrt{p}} = 0.$$

Ahora analicemos el l\'imite de lo que est\'a dentro del segundo par\'entesis
$$\lim_{p \to \infty} \frac{p}{p - 1}
= \lim_{p \to \infty} \frac{1}{1 - \frac{1}{p}}
= \frac{1}{1 - 0} = 1.$$

Como ambos l\'imites existen, por aritm\'etica de l\'imites obtenemos
$$\lim_{p \to \infty} \frac{\ln p}{p (p - 1)} p^\frac{3}{2}
= \left(\lim_{p \to \infty} \frac{\ln p}{\sqrt{p}}\right)\left(\lim_{p \to \infty} \frac{p}{p - 1}\right)
= 0 \cdot 1 = 0.$$

Por definici\'on, dado $\epsilon > 0$ fijo pero arbitrario, existe un $n_0$ a partir del cual
$$\frac{\ln p}{p (p - 1)} p^\frac{3}{2} < \epsilon$$
para todo $p > n_0$. Haciendo $n_1 = \floor{n_0} + 1$, sabemos que para todo $p \geq n_1$
\begin{align*}
\frac{\ln p}{p (p - 1)} &< \frac{\epsilon}{p^\frac{3}{2}} \\
\sum_{p \geq n_1} \frac{\ln p}{p (p - 1)} &< \sum_{p \geq n_1} \frac{\epsilon}{p^\frac{3}{2}}
\end{align*}
y como el menor primo es 2, tenemos que la integral impropia que parte desde 1 supera a la serie
\begin{align*}
\sum_{p \geq n_1} \frac{\ln p}{p (p - 1)} &< \int_{1}^\infty \frac{\epsilon}{x^\frac{3}{2}} dx \\
\sum_{p \geq n_1} \frac{\ln p}{p (p - 1)} &< 2 \epsilon \\
\sum_{p < n_1} \frac{\ln p}{p (p - 1)} + \sum_{p \geq n_1} \frac{\ln p}{p (p - 1)} &< \sum_{p < n_1} \frac{\ln p}{p (p - 1)} + 2 \epsilon \\
\sum_{p = 2}^\infty \frac{\ln p}{p (p - 1)} &< \sum_{p < n_1} \frac{\ln p}{p (p - 1)} + 2 \epsilon.
\end{align*}

Como se trata de una serie positiva, creciente y acotada, concluimos que es convergente.
\end{proof}

\begin{lemma}
Para todo $x \geq 1$ y $p$ un n\'umero primo tenemos
$$\vartheta(x)\ln x + \sum_{p \leq x} \vartheta\left(\frac{x}{p}\right)\ln p = 2x\ln x + O(x).$$
\end{lemma}

\begin{proof}
Analicemos la diferencia de sumatorias utilizando sus definiciones
\begin{align*}
\sum_{n \leq x} \Psi\left(\frac{x}{n}\right)\Lambda(n) - \sum_{p \leq x} \vartheta\left(\frac{x}{p}\right)\ln p
&= \sum_{n \leq x} \sum_{m \leq \frac{x}{n}} \Lambda(m)\Lambda(n) - \sum_{p \leq x} \sum_{q \leq \frac{x}{p}} \ln q \ln p \\
&= \sum_{nm \leq x} \Lambda(n)\Lambda(m) - \sum_{pq \leq x} \ln p \ln q.
\end{align*}

Analicemos que ocurre cuando restamos, la funci\'on de Mangoldt solo actuar\'a sobre las potencias de los primos,
en particular, todas las combinaciones de primos con potencias iguales a uno son exactamente la sumatoria que estamos restando a la derecha.
De esta manera, solo nos queda analizar cuando al menos uno de los exponentes es mayor o igual a dos, y debido a que son primos distintos,
podemos fijar este exponente en el primer n\'umero primo y que el exponente del segundo n\'umero primo tome cualquier valor
\begin{align*}
\sum_{n \leq x} \Psi\left(\frac{x}{n}\right)\Lambda(n) - \sum_{p \leq x} \vartheta\left(\frac{x}{p}\right)\ln p
&= O\left(\sum_{\substack{p^n q^m \leq x \\ n \geq 2, m \geq 1}} \ln p \ln q\right) \\
&= O\left(\sum_{\substack{p^n \leq x \\ n \geq 2}} \ln p \sum_{\substack{q^m \leq \frac{x}{p^n} \\ m \geq 1}} \ln q\right) \\
&= O\left(\sum_{\substack{p^n \leq x \\ n \geq 2}} \ln p \, \Psi\left(\frac{x}{p^n}\right)\right).
\end{align*}

Utilizamos el Lema 5 en la \'ultima igualdad
\begin{align*}
\sum_{n \leq x} \Psi\left(\frac{x}{n}\right)\Lambda(n) - \sum_{p \leq x} \vartheta\left(\frac{x}{p}\right)\ln p
&= O\left(\sum_{\substack{p^n \leq x \\ n \geq 2}} \ln p \, \frac{x}{p^n}\right) \\
&= O\left(x\sum_{\substack{p^n \leq x \\ n \geq 2}} \frac{\ln p}{p^n}\right).
\end{align*}

Para poder desdoblar la sumatoria, tenemos que notar que para que las potencias mayores o iguales a dos de los n\'umeros primos sean menores o iguales a $x$,
como m\'aximo los n\'umeros son menores o iguales que la ra\'iz de $x$, pues de no serlo, estos primos al cuadrado exceder\'ian $x$. Asimismo, reemplazo
los t\'erminos de la sumatoria derecha por una serie geom\'etrica puesto que cada una de ellas converge y acota a la sumatoria finita
\begin{align*}
\sum_{n \leq x} \Psi\left(\frac{x}{n}\right)\Lambda(n) - \sum_{p \leq x} \vartheta\left(\frac{x}{p}\right)\ln p
&= O\left(x \sum_{p \leq \sqrt{x}} \ln p \sum_{n \geq 2} \frac{1}{p^n}\right).
\end{align*}

Ahora realizar\'e el cambio de variable $m = n - 2$ en la sumatoria
\begin{align*}
\sum_{n \leq x} \Psi\left(\frac{x}{n}\right)\Lambda(n) - \sum_{p \leq x} \vartheta\left(\frac{x}{p}\right)\ln p
&= O\left(x \sum_{p \leq \sqrt{x}} \ln p \sum_{m \geq 0} \frac{1}{p^{m + 2}}\right) \\
&= O\left(x \sum_{p \leq \sqrt{x}} \frac{\ln p}{p^2} \sum_{m \geq 0} \frac{1}{p^m}\right) \\
&= O\left(x \sum_{p \leq \sqrt{x}} \frac{\ln p}{p^2} \left(\frac{1}{1 - \frac{1}{p}}\right)\right) \\
&= O\left(x \sum_{p \leq \sqrt{x}} \frac{\ln p}{p^2} \left(\frac{p}{p - 1}\right)\right) \\
&= O\left(x \sum_{p \leq \sqrt{x}} \frac{\ln p}{p (p - 1)}\right).
\end{align*}

Del Lema 12, no solo sabemos que $\sum_{p = 2}^{\infty} \frac{\ln p}{p (p - 1)}$ coverge sino que est\'a acotada por un n\'umero real positivo.
Debido a esto, podemos reducir la expresi\'on
\begin{align*}
\sum_{n \leq x} \Psi\left(\frac{x}{n}\right)\Lambda(n) - \sum_{p \leq x} \vartheta\left(\frac{x}{p}\right)\ln p
&= O\left(x \sum_{p = 2}^\infty \frac{\ln p}{p (p - 1)}\right) \\
&= O(cx) \\
&= O(x).
\end{align*}

Utilicemos el Lema 10 en la sumatoria que involucra $\Psi$ y $\Lambda$
\begin{align*}
\sum_{n \leq x} \Psi\left(\frac{x}{n}\right)\Lambda(n) - \sum_{p \leq x} \vartheta\left(\frac{x}{p}\right)\ln p
&= O(x) \\
2x\ln x + O(x) - \Psi(x)\ln x - \sum_{p \leq x} \vartheta\left(\frac{x}{p}\right)\ln p &= O(x) \\
\Psi(x)\ln x + \sum_{p \leq x} \vartheta\left(\frac{x}{p}\right)\ln p &= 2x\ln x + O(x).
\end{align*}

Finalmente, utilizamos el Lema 11 en $\Psi$ y el Lema 9 en $\ln^2 x$
\begin{align*}
(\vartheta(x) + O(\sqrt{x}\ln x))\ln x + \sum_{p \leq x} \vartheta\left(\frac{x}{p}\right)\ln p &= 2x\ln x + O(x) \\
\vartheta(x)\ln x + O(\sqrt{x}\ln^2 x) + \sum_{p \leq x} \vartheta\left(\frac{x}{p}\right)\ln p &= 2x\ln x + O(x) \\
\vartheta(x)\ln x + O(\sqrt{x}\sqrt{x}) + \sum_{p \leq x} \vartheta\left(\frac{x}{p}\right)\ln p &= 2x\ln x + O(x) \\
\vartheta(x)\ln x + O(x) + \sum_{p \leq x} \vartheta\left(\frac{x}{p}\right)\ln p &= 2x\ln x + O(x) \\
\end{align*}

Concluimos que se cumple $\vartheta(x)\ln x + \sum_{p \leq x} \vartheta\left(\frac{x}{p}\right)\ln p = 2x\ln x + O(x)$ para todo $x \geq 1$.
\end{proof}

\begin{lemma}
La serie
$$\sum_{n = 1}^\infty \frac{1}{n ^ 2}$$
converge.
\end{lemma}

\begin{proof}
Analicemos las sumas parciales telesc\'opicamente para $k \geq 2$
\begin{align*}
\sum_{n = 1}^k \frac{1}{n^2} &= 1 + \sum_{n = 2}^k \frac{1}{n^2} \\
&\leq 1 + \sum_{n = 2}^k \frac{1}{n (n - 1)} \\
&= 1 + \sum_{n = 2}^k \left(\frac{1}{n - 1} - \frac{1}{n}\right) \\
&= 1 + \left(\frac{1}{2 - 1} - \frac{1}{k}\right) \\
&= 2 - \frac{1}{k}.
\end{align*}

Sabemos que $\lim_{k \to \infty} \frac{1}{k} = 0$,
por lo tanto $\sum_{n = 1}^\infty \frac{1}{n ^ 2} < 2$.
Al ser esta una serie creciente, positiva y acotada, esta serie es covergente.
\end{proof}

\begin{theorem}[F\'ormula Asint\'otica de Selberg]
Para todo $x \geq 1$ y $p$, $q$ n\'umeros primos distintos tenemos
$$\sum_{p \leq x} \ln^2 p + \sum_{pq \leq x} \ln p \ln q = 2x\ln x + O(x).$$
\end{theorem}

\begin{proof}
Analicemos la siguiente diferencia
\begin{align*}
\vartheta(x)\ln x - \sum_{p \leq x} \ln^2 p &= \sum_{p \leq x} \ln p \ln x - \sum_{p \leq x} \ln p \ln p \\
&= \sum_{p \leq x} \ln p (\ln x - \ln p) \\
&= \sum_{p \leq x} \ln p \ln \left(\frac{x}{p}\right)
\end{align*}

Tratemos de realizar una mejora al Lema 1. Sabemos que
$$\sum_{n \leq x} \frac{1}{n} = \ln x + \gamma + O\left(\frac{1}{x}\right)$$
y para todo $x \geq 1$, tenemos que $\frac{1}{x} \leq 1$, por lo cual
$$c\frac{1}{x} \leq c \cdot 1$$
obteniendo que $\frac{1}{x} = O(1)$. As\'i, la suma arm\'onica
\begin{align*}
\sum_{n \leq x} \frac{1}{n} &= \ln x + \gamma + O\left(\frac{1}{x}\right) \\
&= \ln x + (\gamma + c_0) \\
&= \ln x + c_1 \\
&= \ln x + O(1) \\
\ln x &= \sum_{n \leq x} \frac{1}{n} + O(1)
\end{align*}
la expresamos de esta manera, mejorando el estimado y ahora podremos utilizarlo a nuestro favor.

Reeplazamos este nuevo estimado en lo anterior
\begin{align*}
\vartheta(x)\ln x - \sum_{p \leq x} \ln^2 p &= \sum_{p \leq x} \ln p \ln \left(\frac{x}{p}\right) \\
&= \sum_{p \leq x} \ln p \left(\sum_{n \leq \frac{x}{p}} \frac{1}{n} + O(1)\right) \\
&= \sum_{p \leq x} \ln p \sum_{n \leq \frac{x}{p}} \frac{1}{n} + O\left(\sum_{p \leq x} \ln p\right) \\
&= \sum_{p \leq x} \sum_{n \leq \frac{x}{p}} \frac{\ln p}{n} + O\left(\sum_{p \leq x} \ln p\right) \\
&= \sum_{n \leq x} \sum_{p \leq \frac{x}{n}} \frac{\ln p}{n} + O\left(\sum_{p \leq x} \ln p\right) \\
&= \sum_{n \leq x} \sum_{p \leq \frac{x}{n}} \frac{\ln p}{n} + O(\vartheta(x)).
\end{align*}

Del Lema 11 tenemos
$$\Psi(x) = \vartheta(x) + O(\sqrt{x}\ln x) = \vartheta(x) + O(\sqrt{x} \ln^2 x),$$
del Lema 9 podemos cambiar la cota de $\ln^2 x$
\begin{align*}
\Psi(x) &= \vartheta(x) + O(\sqrt{x}\sqrt{x}) \\
&= \vartheta(x) + O(x)
\end{align*}
y del Lema 5 tenemos otra cota para $\Psi(x)$
\begin{align*}
\Psi(x) &= \vartheta(x) + O(x) \\
O(x) &= \vartheta(x) + O(x) \\
\vartheta(x) &= O(x).
\end{align*}

Ahora, utilizamos esta cota en la igualdad previa
\begin{align*}
\vartheta(x)\ln x - \sum_{p \leq x} \ln^2 p &= \sum_{n \leq x} \sum_{p \leq \frac{x}{n}} \frac{\ln p}{n} + O(x) \\
&= \sum_{n \leq x} \frac{1}{n}\sum_{p \leq \frac{x}{n}} \ln p + O(x) \\
&= \sum_{n \leq x} \frac{\vartheta\left(\frac{x}{n}\right)}{n} + O(x) \\
&= \sum_{n \leq x} \frac{O\left(\frac{x}{n}\right)}{n} + O(x) \\
&= O\left(x \sum_{n \leq x} \frac{1}{n^2}\right) + O(x).
\end{align*}

Gracias al Lema 14, la sumatoria de inversas al cuadrado est\'a acotada
\begin{align*}
\vartheta(x)\ln x - \sum_{p \leq x} \ln^2 p &= O(2x) + O(x) \\
&= O(x)\\
\vartheta(x)\ln x &= \sum_{p \leq x} \ln^2 p + O(x).
\end{align*}

Finalmente, reemplacemos esto en el Lema 13
\begin{align*}
\vartheta(x)\ln x + \sum_{p \leq x} \vartheta\left(\frac{x}{p}\right)\ln p &= 2x\ln x + O(x) \\
\sum_{p \leq x} \ln^2 p + O(x) + \sum_{p \leq x} \vartheta\left(\frac{x}{p}\right)\ln p &= 2x\ln x + O(x) \\
\sum_{p \leq x} \ln^2 p + O(x) + \sum_{p \leq x} \sum_{q \leq \frac{x}{p}} \ln q \ln p &= 2x\ln x + O(x) \\
\sum_{p \leq x} \ln^2 p + \sum_{pq \leq x} \ln p \ln q &= 2x\ln x + O(x)
\end{align*}
concluyendo con la prueba de la f\'ormula de Selberg.
\end{proof}

\cppfile{formula_de_selberg.cpp}
\end{document}