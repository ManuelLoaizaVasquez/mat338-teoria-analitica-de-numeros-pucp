\documentclass{article}
\usepackage[utf8]{inputenc}
\usepackage{amsfonts,latexsym,amsthm,amssymb,amsmath,amscd,euscript}
\usepackage{mathtools}
\usepackage{framed}
% Descomentar fullpage cuando se quiera utilizar menos margen horizontal
%\usepackage{fullpage}
\usepackage{hyperref}
    \hypersetup{colorlinks=true,citecolor=blue,urlcolor =black,linkbordercolor={1 0 0}}

\newenvironment{statement}[1]{\smallskip\noindent\color[rgb]{1.00,0.00,0.50} {\bf #1.}}{}
\allowdisplaybreaks[1]

\DeclarePairedDelimiter\ceil{\lceil}{\rceil}
\DeclarePairedDelimiter\floor{\lfloor}{\rfloor}

% Comandos para teoremas, definiciones, ejemplos, lemas, etc. para sus respectivos body types.
\renewcommand*{\proofname}{Prueba}
\renewcommand{\contentsname}{Contenido}

\newtheorem{theorem}{Teorema}
\newtheorem*{proposition}{Proposici\'on}
\newtheorem{lemma}[theorem]{Lema}
\newtheorem{corollary}[theorem]{Corolario}
\newtheorem{conjecture}[theorem]{Conjetura}
\newtheorem*{postulate}{Postulado}
\theoremstyle{definition}
\newtheorem{defn}[theorem]{Definici\'on}
\newtheorem{example}[theorem]{Ejemplo}

\theoremstyle{remark}
\newtheorem*{remark}{Observaci\'on}
\newtheorem*{notation}{Notaci\'on}
\newtheorem*{note}{Nota}

% Define tus comandos para hacer la vida m�s f�cil.
\newcommand{\BR}{\mathbb R}
\newcommand{\BC}{\mathbb C}
\newcommand{\BF}{\mathbb F}
\newcommand{\BQ}{\mathbb Q}
\newcommand{\BZ}{\mathbb Z}
\newcommand{\BN}{\mathbb N}

\title{MAT338 Teor\'ia Anal\'itica de N\'umeros}
\author{Manuel Loaiza Vasquez}
\date{Ciclo 2020-1}

\begin{document}

\maketitle

\vspace*{-0.25in}
\centerline{Pontificia Universidad Cat\'olica del Per\'u}
\centerline{Lima, Per\'u}
\centerline{\href{mailto:manuel.loaiza@pucp.edu.pe}{{\tt manuel.loaiza@pucp.edu.pe}}}
\vspace*{0.15in}

\begin{framed}
\'Ultima tarea del curso de T\'opicos de An\'alisis de la Especialidad de Matem\'aticas dictado en la Facultad de Ciencias e Ingenier\'ia en la Pontificia Universidad Cat\'olica del Per\'u (PUCP) por Alfredo Poirier Schmitz en el ciclo 2020-1.
\end{framed}

\begin{statement}{1}
Para todo n\'umero real $x$ mayor o igual a $1$ se cumple la f\'ormula de Selberg
$$\sum_{p \leq x} \ln^2(p) + \sum_{pq \leq x} \ln(p) \ln(q) = 2x\ln(x) + O(x).$$
\end{statement}
 
\begin{lemma}
Para todo $x \geq 1$ tenemos
$$\sum_{n \leq x} \frac{1}{n} = \ln x + \gamma + O\left(\frac{1}{x}\right).$$
\end{lemma}

\begin{lemma}
Para todo $x \geq 1$ tenemos
$$\sum_{n \leq x} \ln n = x\ln x - x + O(\ln x).$$
\end{lemma}

\begin{lemma}
Para toda funci\'on aritm\'etia $f$ se cumple
$$\sum_{n \leq x} \sum_{d \mid n} f(d) = \sum_{n \leq x} f(n) \floor{\frac{x}{n}}.$$
\end{lemma}

\begin{lemma}
Para todo n\'umero real $x$ tenemos
$$\floor{x} = x + O(1).$$
\end{lemma}

\begin{proof}
Sea $x = n + r$ un n\'umero real no negativo, con $n \in \BZ$ y $0 \leq r < 1$. De esta manera, por definici\'on de m\'aximo entero tenemos
\begin{align*}
\floor{x} &= n \\
&= x - r \\
&= x + O(1).
\end{align*}
Concluimos trivialmente que $\floor{x} = x + O(1)$.
\end{proof}

\begin{lemma}
Para todo $x \geq 1$ tenemos
$$\Psi(x) = O(x).$$
\end{lemma}

\begin{lemma}
La funci\'on de Mangoldt se puede expresar como el siguiente producto de Dirichlet
$$\Lambda = \mu * \ln.$$
\end{lemma}

\begin{proof}
Sabemos que $\Lambda * 1 = \ln$ y aplicamos los propiedades del producto de Dirichlet
\begin{align*}
(\Lambda * 1) * \mu &= \ln * \mu \\
\Lambda * (1 * \mu) &= \ln * \mu \\
\Lambda * \mathbb{U} &= \ln * \mu \\
\Lambda &= \ln * \mu \\
\Lambda &= \mu * \ln.
\end{align*}
Concluimos que $\Lambda = \mu * \ln$.
\end{proof}

\begin{lemma}
Para todo $x \geq 1$ tenemos
$$\sum_{n \leq x} \frac{\Lambda(n)}{n} = \ln x + O(1).$$
\end{lemma}

\begin{proof}
Sabemos que $\ln = \Lambda * 1$, por lo tanto, desarrollamos
\begin{align*}
\ln n &= \sum_{d \mid n} \Lambda(d) \\
\sum_{n \leq x} \ln n &= \sum_{n \leq x} \sum_{d \mid n} \Lambda(d).
\end{align*}
Del Lema 3 obtenemos
\begin{align*}
\sum_{n \leq x} \ln n &= \sum_{n \leq x} \Lambda(n) \floor{\frac{x}{n}}
\end{align*}
y utilizando el Lema 4 conseguimos
\begin{align*}
\sum_{n \leq x} \ln n &= \sum_{n \leq x} \Lambda(n) \left(\frac{x}{n} + O(1)\right) \\
&= x \sum_{n \leq x} \frac{\Lambda(n)}{n} + O\left(\sum_{n \leq x} \Lambda(x) \right) \\
&= x \sum_{n \leq x} \frac{\Lambda(n)}{n} + O(\Psi(x)).
\end{align*}
La astucia de Chebyshev es la que permite mejorar el estimado que obtuvimos en clase de 
$\Psi(x) = \vartheta(x) + O(\sqrt{x} \ln^2 x)$ con $\vartheta(x) = O(x \ln x)$ 
por $\Psi(x) = O(x)$ gracias al Lema 5.
Utilizando este poderoso teorema de Chebyshev y aplicando el Lema 2 en la suma acumulada de logaritmos conseguimos
\begin{align*}
x\ln x - x + O(\ln x) &= x \sum_{n \leq x} \frac{\Lambda(n)}{n} + O(x)
\end{align*}
y despejamos
\begin{align*}
\sum_{n \leq x} \frac{\Lambda(n)}{n} &= \frac{x \ln x}{x} - 1 + O\left(\frac{\ln x}{x}\right) + O(1) \\
&= \ln x - 1 + O(1) + O\left(\frac{\ln x}{x}\right) \\
&\leq \ln x - 1 + c_0 + O(1) \\
&\leq \ln x + c_1 \\
&= \ln x + O(1).
\end{align*}
Concluimos que $\sum_{n \leq x} \frac{\Lambda(n)}{n} = \ln x + O(1)$.
\end{proof}

\begin{lemma}
Para todo $f, g : [1, \infty) \to \BR$ con $g(x) = \sum_{n \leq x} f\left(\frac{x}{n}\right) \ln x$ tenemos
$$\sum_{n \leq x} \mu(n) g\left(\frac{x}{n}\right) = f(x) \ln(x) + \sum_{n \leq x} f\left(\frac{x}{n}\right) \Lambda(n).$$
\end{lemma}

\begin{proof}
Desarrollemos la sumatoria que queremos analizar
\begin{align*}
\sum_{n \leq x} \mu(n) g\left(\frac{x}{n}\right) &= \sum_{cd \leq x} \mu(c)g(d) \\
&= \sum_{cd \leq x} \mu(c) \sum_{e \leq d} f\left(\frac{d}{e}\right) \ln(d) \\
&= \sum_{cd \leq x} \mu(c) \ln\left(\frac{x}{c}\right) f \left(\frac{x}{cd}\right) \\
&= \sum_{n \leq x} f\left(\frac{x}{n}\right) \sum_{d \mid n} \mu(d) \ln\left(\frac{x}{n}\right) \\
&= \sum_{n \leq x} f\left(\frac{x}{n}\right) \sum_{d \mid n} \mu(d) \left[\ln \left(\frac{x}{n}\right) + \ln\left(\frac{n}{d}\right)\right] \\
&= \left[\sum_{n \leq x} f\left(\frac{x}{n}\right) \ln\left(\frac{x}{n}\right) \sum_{d \mid n} \mu(d)\right]
+  \left[\sum_{n \leq x} f\left(\frac{x}{n}\right) \sum_{d \mid n} \mu(d) \ln\left(\frac{n}{d}\right)\right].
\end{align*}
Asimismo, sabemos que $\mu * 1 = \mathbb{U}$ y
sabemos que el \'unico valor de $\mathbb{U}$ distinto de cero se obtiene en uno.
Por consiguiente, la igualdad se reduce a
\begin{align*}
\sum_{n \leq x} \mu(n)g\left(\frac{x}{n}\right) &= f(x)\ln x + \sum_{n \leq x} f\left(\frac{x}{n}\right) (\mu * \ln)(n).
\end{align*}
Finalmente, utilizamos el Lema $6$ para concluir que $\sum_{n \leq x} \mu(n)g\left(\frac{x}{n}\right) = f(x)\ln x + \sum_{n \leq x} f\left(\frac{x}{n}\right)\Lambda(n)$.
\end{proof}

\begin{lemma}
Para todo $x \geq 1$ tenemos
$$\ln^2 x = O(\sqrt{x}).$$
\end{lemma}

\begin{proof}
Sabemos que para todo $x \geq 1$ se cumple que $x > \ln x$ analizando simplemente la derivada de $x - \ln x$.
\begin{align*}
\ln^2 x &= \ln^2 ((x ^ {\frac{1}{4}}) ^ 4) \\
&= 16 \ln^2 (x ^ {\frac{1}{4}}) \\
&< 16 (x ^ {\frac{1}{4}}) ^ 2 \\
&= 16 \sqrt{x}.
\end{align*}
Por lo tanto, concluimos que $\ln^2 x = O(\sqrt{x})$.
\end{proof}

\end{document}