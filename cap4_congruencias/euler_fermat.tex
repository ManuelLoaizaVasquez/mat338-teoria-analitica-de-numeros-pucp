\documentclass[main.tex]{subfiles}

\begin{document}
\begin{theorem}[Teorema de Euler]
    Para todo $a \in \BZ_n^*$ se tiene lo siguiente
    $$a^{\varphi(m)}\equiv 1\pmod{m}$$
\end{theorem}

\begin{proof}
    Multiplicar\'e los $\varphi(m)$ elementos de $\BZ_m^*$ de dos maneras distintas y ambas son id\'enticas, pues sabemos que la multiplicaci\'on por $a$ es una biyecci\'on, lo cual implica que las colecciones $\BZ_m^*$ y $a\cdot\BZ_m^*$ son id\'enticas.
    $$\prod_{x \in \BZ_m^*} x \equiv \prod_{x \in \BZ_m^*} (ax) \equiv a^{\varphi(m)}\prod_{x \in \BZ_m^*}x \pmod{m}$$
    Adem\'as, el producto de todos los n\'umeros relativamente primos con $m$ sigue siendo relativamente primo con $m$, por lo cual multiplicaremos por su inversa
    \begin{align*}
        a^{\varphi(m)}\prod_{x\in\BZ_m^*}x&\equiv\prod_{x\in\BZ_m^*}x\pmod{m}\\
        a^{\varphi(m)}&\equiv1\pmod{m}
    \end{align*}
    y obtenemos la relaci\'on buscada.
\end{proof}

\begin{corollary}[Pequeño Teorema de Fermat]
    Si $p$ es primo, para todo $a \in \BZ_p^*$ entonces se tiene lo siguiente
    $$a^p\equiv a\pmod{p}$$
\end{corollary}

\begin{proof}
    Como $a < p$ y $p$ es primo, entonces $(a, p) = 1$. Utilizando el teorema de Euler tenemos lo siguiente
    \begin{align*}
        a^{\varphi(p)}\equiv1\pmod{p}\\
        a^{p-1}\equiv1\pmod{p}\\
        a^p\equiv a\pmod{p}
    \end{align*}
    El cual vendr\'ia a ser un caso particular del teorema anterior.
\end{proof}

\begin{example}
    ¿Cu\'anto vale $7^{111}$ m\'odulo 13?
\end{example}

\begin{proof}
    Lo primero que observamos es que $(7, 13) = 1$, por lo tanto podemos utilizar el teorema de Euler
    \begin{align*}
        7^{\varphi(13)}&\equiv 1\pmod{13}\\
        7^{12}&\equiv 1\pmod{13}\\
        (7^{12})^9 &\equiv 1\pmod{13}\\
        7^{108}\cdot 7^3 &\equiv 7^3\pmod{13}\\
        7^{111} &\equiv 5\pmod{13} 
    \end{align*}
\end{proof}

\begin{example}
    ¿En qu\'e digito termina $3^{100}$?
\end{example}

\begin{proof}
    Lo primero que observamos es que $(3, 10) = 1$, por lo tanto podemos utilizar el teorema de Euler
    \begin{align*}
        3^{\varphi(10)}&\equiv 1\pmod{10}\\
        3^4&\equiv 1\pmod{10}\\
        (3^4)^{25}&\equiv 1\pmod{10}\\
        3^{100}&\equiv 1\pmod{10}
    \end{align*}
    Concluimos que $3^{100}$ termina en $1$, pues el \'ultimo d\'igito es el residuo al dividir el n\'umero entre $10$.
\end{proof}
\end{document}