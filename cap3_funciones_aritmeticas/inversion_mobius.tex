\documentclass[main.tex]{subfiles}

\begin{document}
Sabemos que $\sum_{d \mid n}\mu(d) = \mathbb U(n)$. As\'i, observamos que $\mu * 1 = \mathbb U$. En otras palabras, la funci\'on uno y la funci\'on de M\"obius son inversas una de la otra.
$$1^{-1}=\mu \quad \mu^{-1}=1$$
\begin{theorem}[F\'ormula de Inversi\'on de M\"obius]
Se satisface la ecuaci\'on
$$f(n) = \sum_{d \mid n}g(d)$$
si y solo si
$$g(n) = \sum_{d \mid n} f(d)\mu(\frac{n}{d})$$
\end{theorem}
\begin{proof}
$\implies$: Tenemos $f = g * 1$.
\begin{align*}
    f * 1^{-1} &= (g * 1) * 1^{-1}\\
    f * \mu &= g * (1 * \mu)\\
    f * \mu &= g * \mathbb U\\
    g &= f * \mu
\end{align*}
$\impliedby$: Tenemos $g = f * \mu$.
\begin{align*}
    g * 1 &= (f * \mu) * 1\\
    g * 1 &= f * (\mu * 1)\\
    g * 1 &= f * \mathbb U\\
    f &= g * 1
\end{align*}
\end{proof}
\begin{theorem}
$\varphi$ y $\mu$ se relacionan de la siguiente manera
$$\varphi(n) = n\sum_{d \mid n}\frac{\mu(d)}{d}$$
\end{theorem}
\begin{proof}
Sabemos que $\sum_{d \mid n}\varphi(d) = n$. As\'i, observamos que $\varphi * 1 = N$.
\begin{align*}
    (\varphi * 1) * \mu &= N * \mu\\
    \varphi &= \mu * N\\
    \varphi(n) &= \sum_{d \mid n} \mu(d)\cdot(\frac{n}{d})\\
    \varphi(n) &= n\sum_{d \mid n} \frac{\mu(d)}{d}
\end{align*}
\end{proof}
\begin{theorem}
Sea $p$ un n\'umero primo y $\alpha \geq 1$ se cumple
$$\varphi(p^\alpha) = p^{\alpha}\left(1 - \frac{1}{p}\right)$$
\end{theorem}
\begin{proof}
Utilizando el teorema anterior, tenemos
\begin{align*}
    \varphi(p^\alpha) &= p^\alpha\sum_{d \mid n}\frac{\mu(d)}{d}\\
    &= p^\alpha\left(\frac{\mu(1)}{1} + \frac{\mu(p)}{p} + \dots + \frac{\mu(p^\alpha)}{p^\alpha}\right)\\
    &= p^\alpha\left(\frac{\mu(1)}{1} + \frac{\mu(p)}{p}\right)\\
    &= p^\alpha\left(1 - \frac{1}{p}\right)
\end{align*}
\end{proof}
\begin{theorem}
Sea $n \geq 1$, tenemos
$$\varphi(n) = n \prod_{p | n}\left(1 - \frac{1}{p}\right)$$
\end{theorem}
\begin{proof}
El caso $n = 1$ es trivial. Para $n > 1$, por el teorema $3.19$.
$$\varphi(n) = n \sum_{d|n} \frac{\mu(d)}{d}$$
Sabemos que $\mu$ eliminar\'a todos los divisores que no son libres de cuadrados, por lo que nos quedar\'iamos con la suma de los divisores con primos distintos. Sea $n = p_1^{ord_{p_1}(n)}\dots p_k^{ord_{p_k}(n)}$. Desarrollando la sumatoria
\begin{align*}
    \varphi(n) &= n\left(1 + \frac{\mu(p_1)}{p_1} + \dots + \frac{\mu(p_k)}{p_k} + \frac{\mu(p_1 p_2)}{p_1 p_2} + \dots + \frac{\mu(p_{k-1} p_k)}{p_{k-1} p_k} + \dots + \frac{\mu(p_1 \dots p_k)}{p_1 \dots p_k}\right)\\
    &= n\left(1 - \sum\frac{1}{p_i} + \sum\frac{1}{p_i p_j} + \dots + (-1)^k\sum\frac{1}{p_1 \dots p_k}\right)\\
    &= n\prod_{p \mid n}\left(1 - \frac{1}{p}\right)
\end{align*}
La \'ultima igualdad se puede probar sencillamente con inducci\'on.
\end{proof}
\end{document}