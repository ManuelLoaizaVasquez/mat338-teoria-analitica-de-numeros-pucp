\documentclass[main.tex]{subfiles}
\begin{document}
\begin{defn}
Una funci\'on aritm\'etica $f$ es multiplicativa si $f$ no es id\'entica a cero y si
$$f(mn) = f(m)f(n),\quad (m, n) = 1$$
Una funci\'on multiplicativa $f$ ser\'a llamada totalmente multiplicativa cuando
$$f(mn) = f(m)f(n),\quad\forall m, n$$
\end{defn}
\begin{example}
Las funciones aritm\'eticas $0, 1, \mathbb U$ son totalmente multiplicativas.
\end{example}

\begin{example}
$\mu$ y $\varphi$ son funciones aritm\'eticas multiplicativas. En el teorema 3.7 demostramos que $\mu$ es multiplicativa. Nos falta probar de que $\varphi$ lo es.
\end{example}
\begin{theorem}
Dados $M, N$ enteros no nulos tales que $(M, N)=1$ se cumple
$$\varphi(MN) = \varphi(M)\varphi(N)$$
\end{theorem}
\begin{proof}
Como $M$ y $N$ son coprimos, entonces puedo expresarlos como
$$M = p_1^{\alpha_1} \dots p_m^{\alpha_m}$$
$$N = q_1^{\beta_1} \dots q_n^{\beta_n}$$
As\'i tenemos que
\begin{align*}
    \varphi(MN) &= MN \prod_{r \mid MN} (1 - \frac{1}{r})\\
    &= M\prod_{p \mid M}(1 - \frac{1}{p})N\prod_{q \mid N}(1 - \frac{1}{q})\\
    &= \varphi(M)\varphi(N)
\end{align*}
Concluyendo de que se cumple la igualdad.
\end{proof}
\begin{theorem}
Sea $f$ con $f(1)=1$, entonces, para todo $p_i, p$ primos y $\alpha_i, \alpha \geq 1$
\begin{enumerate}
    \item $f$ es multiplicativa si y solo si $f(p_1^{\alpha_1}\dots p_k^{\alpha_k})=f(p_1^{\alpha_1})\dots f(p_k^{\alpha_k})$
    \item Si $f$ es multiplicativa, entonces $f$ es totalmente multiplicativa si solo si $f(p^\alpha)=f(p)^\alpha$
\end{enumerate}
\end{theorem}
\begin{proof}
\end{proof}
Una manera sencilla de construir funciones multiplicativas es centr\'andose en un primo y sus potencias. Esto lo podemos hacer al despreciar la contribuci\'on de los otros primos, es decir, ponemos $f(n) = 0$ si $n$ no es potencia de $p$. A\'un falta asignar los valores correspondientes a $f(p^\alpha) = \pi_\alpha$, sin olvidar de que $\pi_0 = 1$. Luego de esto no hay otra restricci\'on que no sea nuestra imaginaci\'on. Si queremos que esta funci\'on adem\'as sea totalmente multiplicativa, la \'unica exigencia adicional ser\'a que $\pi_\alpha = \pi_1^\alpha$ debido a la propiedad $f(p^\alpha) = f(p)^\alpha$.
\begin{theorem}
Si $f$ y $g$ son multiplicativas, entonces $f*g$ tambi\'en lo es.
\end{theorem}
\begin{proof}
Sea $h = f*g$ y $m, n \in \BZ$ tal que $(m, n) = 1$. Luego
$$h(mn) = \sum_{c \mid mn} f(c)g(\frac{mn}{c})$$
Asimismo, podemos expresar a cada divisor $c$ de $mn$ como $c = ab$ donde $a \mid m$ y $b \mid n$. Es m\'as, $(a, b) = 1$, $(\frac{m}{a}, \frac{n}{b}) = 1$. As\'i
\begin{align*}
    h(mn) &= \sum_{a \mid m, b \mid n} f(ab)g(\frac{m}{a}\frac{n}{b})\\
    &= \sum_{a \mid m, b \mid n} f(a)f(b)g(\frac{m}{a})g(\frac{n}{b})\\
    &= \sum_{a \mid m, b \mid n} f(a)g(\frac{m}{a})f(b)g(\frac{n}{b})\\
    &= \sum_{a \mid m}f(a)g(\frac{m}{a})\sum_{b \mid n}f(b)g(\frac{n}{b})\\
    &= h(m)h(n)
\end{align*}
Concluimos con la prueba.
\end{proof}
\begin{theorem}
Si $g$ y $f*g$ son multiplicativas, entonces $f$ es multiplicativa.
\end{theorem}
\begin{proof}
\end{proof}
\begin{theorem}
Si $g$ es multiplicativa, entonces $g^{-1}$ es multiplicativa.
\end{theorem}
\begin{proof}
\end{proof}
\begin{theorem}
Sea $f$ multiplicativa. $f$ es totalmente multiplicativa si y solo si
$$f^{-1}(n) = \mu(n)f(n),\;\forall n \geq 1$$
\end{theorem}
\begin{proof}
\end{proof}

\end{document}