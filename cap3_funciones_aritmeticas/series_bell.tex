\documentclass[main.tex]{subfiles}
\begin{document}
\begin{defn}
Sea $f$ una funci\'on aritm\'etica y $p$ un n\'umero primo, denotamos a $f_p(x)$ a la serie de potencia
$$f_p(x) = \sum_{n = 0}^\infty f(p^n)x^n$$
y esta es llamada serie de Bell de $f$ m\'odulo $p$.
\end{defn}
A diferencia de las series de an\'alisis, ac\'a entran a tallar apenas aspectos manipulativos formales; es decir, nos desentendemos de asuntos relacionados con la convergencia o no de estos objetos.
\begin{theorem}
Sean $f$ y $g$ funciones multiplicativas, luego $f = g$ si y solo si
$$f_p(x) = g_p(x),\;\forall p \text{ primo}$$
\end{theorem}
\begin{proof}
\end{proof}
\begin{example}
Para la funci\'on $1$ y $|x| < 1$, tenemos
$$1_p(x) = \sum_{n = 0}^\infty1(p^n)x^n = \sum_{n=0}^\infty x^n = \frac{1}{1-x}$$
\end{example}
\begin{example}
Para la funci\'on $\mathbb U$ tenemos
$$\mathbb U_p(x) = \sum_{n = 0}^\infty \mathbb U(p^n)x^n = 1 + \sum_{n = 1}^\infty \mathbb U(p^n)x^n = 1$$
\end{example}
\begin{example}
Para la funci\'on $\mu$ de M\"obius tenemos
$$\mu_p(x) = \mu(1) + \mu(p)x = 1 - x$$
\end{example}
\begin{example}
Para la funci\'on $\varphi$ de Euler y $|x| < 1$ tenemos
\begin{align*}
    \sum_{k = 0}^n \varphi(p^k)x^k &= \varphi(1) + \varphi(p)x + \dots + \varphi(p^k)x^k\\
    &= 1 + (p-1)x + \dots + p^{n-1}(p-1)x^n\\
    &= (1 + px + \dots + p^n x^n) - (x + \dots + p^{n-1}x^n)\\
    \lim_{n\to\infty}\sum_{k=0}^n \varphi(p^k)x^k &= \frac{1}{1 - px} - \frac{x}{1-px}\\
    \varphi_p(x) &= \frac{1 - x}{1 - px}
\end{align*}
\end{example}
\begin{example}
Funciones completamente multiplicativas. Si $f$ es completamente multiplicativa, luego $f(p^n) = f(p)^n$, para todo $n \geq 0$, por lo tanto, la serie de Bell es una serie geom\'etrica
$$f_p(x) = \sum_{n = 0}^\infty f(p)^n x^n = \frac{1}{1 - f(p)x}$$
\end{example}
\begin{theorem}
Para cualquier par de funciones aritm\'eticas $f, g$, sea $h = f * g$. Luego, para todo primo $p$ tenemos
$$h_p(x) = f_p(x)g_p(x)$$
\end{theorem}
\begin{proof}
\end{proof}
\end{document}