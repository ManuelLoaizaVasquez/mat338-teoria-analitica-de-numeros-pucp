\documentclass{article}
\usepackage[utf8]{inputenc}
\usepackage{amsfonts,latexsym,amsthm,amssymb,amsmath,amscd,euscript, mathtools}
\usepackage{framed}
%\usepackage{fullpage}
\usepackage{hyperref}
    \hypersetup{colorlinks=true,citecolor=blue,urlcolor =black,linkbordercolor={1 0 0}}

%Below are the theorem, definition, example, lemma, etc. body types.
\renewcommand*{\proofname}{Prueba}
\renewcommand{\contentsname}{Contenido}

\newtheorem{theorem}{Teorema}[section]
\newtheorem{proposition}[theorem]{Proposici\'on}
\newtheorem{lemma}[theorem]{Lema}
\newtheorem{corollary}[theorem]{Corolario}
\newtheorem{conjecture}[theorem]{Conjetura}
\newtheorem{postulate}[theorem]{Postulado}
\theoremstyle{definition}
\newtheorem{defn}[theorem]{Definici\'on}
\newtheorem{example}[theorem]{Ejemplo}

\theoremstyle{remark}
\newtheorem*{remark}{Observaci\'on}
\newtheorem*{notation}{Notaci\'on}
\newtheorem*{note}{Nota}

% You can define new commands to make your life easier.
\newcommand{\BR}{\mathbb R}
\newcommand{\BC}{\mathbb C}
\newcommand{\BF}{\mathbb F}
\newcommand{\BQ}{\mathbb Q}
\newcommand{\BZ}{\mathbb Z}
\newcommand{\BN}{\mathbb N}

\DeclareMathOperator{\ord}{ord}

\usepackage{subfiles}

\title{MAT338 Teor\'ia Anal\'itica de N\'umeros} % IMPORTANT: Change the problemset number as needed.
\author{Manuel Loaiza Vasquez}
\date{Ciclo 2020--1}

\begin{document}

\maketitle

\vspace*{-0.25in}
% Just so that your CA's can come knocking on your door when you don't hand in that problemset on time...
\centerline{Pontificia Universidad Cat\'olica del Per\'u}
\centerline{Lima, Per\'u}
\centerline{\href{mailto:manuel.loaiza@pucp.edu.pe}{{\tt manuel.loaiza@pucp.edu.pe}}}
\vspace*{0.15in}

\begin{framed}
    Notas de clase del curso de T\'opicos de An\'alisis de la Especialidad de Matem\'aticas dictado en la Facultad de Ciencias e Ingenier\'ia en la Pontificia Universidad Cat\'olica del Per\'u (PUCP) por el PhD. Alfredo Poirier Schmitz en el ciclo 2020 - 1. Algunas secciones han sido extra\'idas de los libros {\it Introduction to Analytic Number Theory} de Tom M. Apostol, {\it A Classical Introduction to Modern Number Theory} de Kennet Ireland y Michael Rosen y {\it An Introduction to the Theory of Numbers} de Ivan Niven y Herbert Zuckerman. Si tiene alguna duda o encuentra alg\'un error no dude en contactarme v\'ia correo electr\'onico.
\end{framed}

\tableofcontents

%%%%%%%%%%%%%%%%%%%%%%%%%%%%%%%%%%%%%%%%%%%%%%%%%%
\newpage
\section{Teor\'ia de Conjuntos}
\subsection{El lenguaje de la teor\'ia de conjuntos}
\subfile{cap1_teoria_conjuntos/conjuntos}

\subsection{Conjuntos finitos e infinitos}
\subfile{cap1_teoria_conjuntos/finitos_infinitos}

\subsection{Los n\'umeros naturales}
\subfile{cap1_teoria_conjuntos/naturales}

\subsection{Conjuntos enumerables}
\subfile{cap1_teoria_conjuntos/enumerables}

\subsection{Los n\'umeros enteros}
\subfile{cap1_teoria_conjuntos/enteros}

%%%%%%%%%%%%%%%%%%%%%%%%%%%%%%%%%%%%%%%%%%%%%%%%%%
\newpage
\section{El Teorema Fundamental de la Aritm\'etica}
En este cap\'itulo se introducir\'an los conceptos b\'asicos de la teor\'ia de n\'umeros elemental como la divisibilidad, m\'aximo com\'un divisor, primalidad y factorizaci\'on de un n\'umero. A diferencia del colegio, haremos uso del \'algebra abstracta, en particular, de los anillos e ideales para realizar la mayor\'ia de pruebas y extender las definiciones.

\subsection{Divisibilidad e Ideales}
\subfile{cap2_aritmetica/divisibilidad_ideales}

\subsection{Primos}
\subfile{cap2_aritmetica/primos}

\subsection{Factorizaci\'on \'Unica}
\subfile{cap2_aritmetica/factorizacion_unica}

\subsection{M\'aximo Com\'un Divisor}
\subfile{cap2_aritmetica/maximo_comun_divisor}

%%%%%%%%%%%%%%%%%%%%%%%%%%%%%%%%%%%%%%%%%%%%%%%%%%
\newpage
\section{Funciones Aritm\'eticas}
En la Teor\'ia de N\'umeros, al igual que en otras ramas de las Matem\'aticas, hay una preocupaci\'on por las sucesiones de n\'umeros reales o complejos.

\begin{defn}
    Una funci\'on definida en los enteros positivos con valores reales o complejos es llamada una funci\'on aritm\'etica.
\end{defn}

\subsection{La Funci\'on de M\"obius}
\subfile{cap3_funciones_aritmeticas/funcion_de_mobius}

\subsection{La Funci\'on Indicatriz de Euler}
\subfile{cap3_funciones_aritmeticas/funcion_de_euler}

\subsection{El Producto de Dirichlet}
\subfile{cap3_funciones_aritmeticas/producto_dirichlet}

\subsection{Inversi\'on de M\"obius}
\subfile{cap3_funciones_aritmeticas/inversion_mobius}

\subsection{Inversa de Dirichlet}
\subfile{cap3_funciones_aritmeticas/inversa_dirichlet}

\subsection{La Funci\'on de Mangoldt}
\subfile{cap3_funciones_aritmeticas/funcion_mangoldt}

\subsection{La Funci\'on de Liouville}
\subfile{cap3_funciones_aritmeticas/funcion_liouville}

\subsection{Funciones Multiplicativas}
\subfile{cap3_funciones_aritmeticas/funciones_multiplicativas}

\subsection{Series de Bell}
\subfile{cap3_funciones_aritmeticas/series_bell}
%%%%%%%%%%%%%%%%%%%%%%%%%%%%%%%%%%%%%%%%%%%%%%%%%%
\newpage
\section{Congruencias}
\subsection{Definici\'on y propiedades en $\BZ$}
\subfile{cap4_congruencias/congruencias}

\subsection{Teorema de Euler-Fermat}
\subfile{cap4_congruencias/euler_fermat}

\subsection{Teorema del Resto Chino}
\subfile{cap4_congruencias/resto_chino}
\end{document}